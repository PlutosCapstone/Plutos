\documentclass{article}

\usepackage{tabularx}
\usepackage{booktabs}
\usepackage{float}
\usepackage{amssymb}

\title{Reflection and Traceability Report on \progname}

\author{\authname}

\date{}

%% Comments

\usepackage{color}

\newif\ifcomments\commentstrue %displays comments
%\newif\ifcomments\commentsfalse %so that comments do not display

\ifcomments
\newcommand{\authornote}[3]{\textcolor{#1}{[#3 ---#2]}}
\newcommand{\todo}[1]{\textcolor{red}{[TODO: #1]}}
\else
\newcommand{\authornote}[3]{}
\newcommand{\todo}[1]{}
\fi

\newcommand{\wss}[1]{\authornote{blue}{SS}{#1}} 
\newcommand{\plt}[1]{\authornote{magenta}{TPLT}{#1}} %For explanation of the template
\newcommand{\an}[1]{\authornote{cyan}{Author}{#1}}

%% Common Parts

\newcommand{\progname}{ProgName} % PUT YOUR PROGRAM NAME HERE
\newcommand{\authname}{Team \#, Team Name
\\ Student 1 name
\\ Student 2 name
\\ Student 3 name
\\ Student 4 name} % AUTHOR NAMES                  

\usepackage{hyperref}
    \hypersetup{colorlinks=true, linkcolor=blue, citecolor=blue, filecolor=blue,
                urlcolor=blue, unicode=false}
    \urlstyle{same}
                                


\begin{document}

\maketitle

\plt{Reflection is an important component of getting the full benefits from a
learning experience.  Besides the intrinsic benefits of reflection, this
document will be used to help the TAs grade how well your team responded to
feedback.  Therefore, traceability between Revision 0 and Revision 1 is and
important part of the reflection exercise.  In addition, several CEAB (Canadian
Engineering Accreditation Board) Learning Outcomes (LOs) will be assessed based
on your reflections.}

\section{Changes in Response to Feedback}

\plt{Summarize the changes made over the course of the project in response to
feedback from TAs, the instructor, teammates, other teams, the project
supervisor (if present), and from user testers.}

\plt{For those teams with an external supervisor, please highlight how the feedback 
from the supervisor shaped your project.  In particular, you should highlight the 
supervisor's response to your Rev 0 demonstration to them.}

\plt{Version control can make the summary relatively easy, if you used issues
and meaningful commits.  If you feedback is in an issue, and you responded in
the issue tracker, you can point to the issue as part of explaining your
changes.  If addressing the issue required changes to code or documentation, you
can point to the specific commit that made the changes.  Although the links are
helpful for the details, you should include a label for each item of feedback so
that the reader has an idea of what each item is about without the need to click
on everything to find out.}

\plt{If you were not organized with your commits, traceability between feedback
and commits will not be feasible to capture after the fact.  You will instead
need to spend time writing down a summary of the changes made in response to
each item of feedback.}

\plt{You should address EVERY item of feedback.  A table or itemized list is
recommended.  You should record every item of feedback, along with the source of
that feedback and the change you made in response to that feedback.  The
response can be a change to your documentation, code, or development process.
The response can also be the reason why no changes were made in response to the
feedback.  To make this information manageable, you will record the feedback and
response separately for each deliverable in the sections that follow.}

\plt{If the feedback is general or incomplete, the TA (or instructor) will not
be able to grade your response to feedback.  In that case your grade on this
document, and likely the Revision 1 versions of the other documents will be
low.} 

\subsection{SRS and Hazard Analysis}

\begin{table}[H]
    \caption{SRS Issues}
    \noindent 
    \makebox[\textwidth]{%
    \begin{tabular}{p{1.5cm} p{2cm} p{3.5cm} c c p{4cm}} 
    \toprule 
        \textbf{Issue Number} & \textbf{Source} & \textbf{Issue Title} &
        \textbf{Addressed?} & \textbf{PR} & \textbf{Comments} \\ 
        \midrule
        \href{https://github.com/PlutosCapstone/Plutos/issues/106}{106} & TA &
        docs(SRS): Add section for "normal operation" &  &
        \href{https://github.com/PlutosCapstone/Plutos/pull/x}{x} &   \\ \hline
        \href{https://github.com/PlutosCapstone/Plutos/issues/107}{107} & TA &
        docs(SRS): Add section for undesired event handling &  &
        \href{https://github.com/PlutosCapstone/Plutos/pull/x}{x} &   \\ \hline
        \href{https://github.com/PlutosCapstone/Plutos/issues/108}{108} & TA &
        docs(SRS): Add fit criterions to requirements & $\checkmark$ &
        \href{https://github.com/PlutosCapstone/Plutos/pull/x}{310} & Changes made  \\ \hline
        \href{https://github.com/PlutosCapstone/Plutos/issues/109}{109} & TA &
        docs(SRS): Link requirements to rationale section & $\checkmark$ &
        \href{https://github.com/PlutosCapstone/Plutos/pull/303}{303} & Linked
        requirements to rationale section  \\ \hline
        \href{https://github.com/PlutosCapstone/Plutos/issues/110}{110} & TA &
        docs(SRS): Link likely and unlikely changes to requirements &
        $\checkmark$ &
        \href{https://github.com/PlutosCapstone/Plutos/pull/303}{303} & Linked
        requirements to un/likely changes  \\\hline
        
        % PEER REVIEW 
        \href{https://github.com/PlutosCapstone/Plutos/issues/38}{38} & Peer
        Review & peer-review[team 23]: OCR system accuracy & $\checkmark$ & $-$
        & This was a question, which was answered as a comment in the issue.  \\ \hline
        \href{https://github.com/PlutosCapstone/Plutos/issues/39}{39} & Peer
        Review & peer-review[team 23]: OCR pipeline privacy & $\checkmark$ & $-$ &
        This was a question, which was answered as a comment in the issue.  \\ \hline
        \href{https://github.com/PlutosCapstone/Plutos/issues/40}{40} & Peer
        Review & peer-review[team 23]: release platform & $\checkmark$
        & \href{https://github.com/PlutosCapstone/Plutos/pull/x}{310} & Changes made \\ \hline
        \href{https://github.com/PlutosCapstone/Plutos/issues/41}{41} & Peer
        Review & peer-review[team23]: Non-Functional Requirements Verifiability
        & $\checkmark$ & \href{https://github.com/PlutosCapstone/Plutos/pull/179}{179} &
        Changes made  \\ \hline
        \href{https://github.com/PlutosCapstone/Plutos/issues/42}{42} & Peer
        Review & peer-review[team 23]: Ambiguous Response Time Specifications  &
        $\checkmark$ &
        \href{https://github.com/PlutosCapstone/Plutos/pull/179}{179} & Changes
        made  \\ \hline
        \href{https://github.com/PlutosCapstone/Plutos/issues/43}{43} & Peer
        Review & peer-review[team 23]: Item Recognition and Categorization
        Requirements  & $\checkmark$ &
        \href{https://github.com/PlutosCapstone/Plutos/pull/299}{299} & Slightly
        reworded requirements to make them more clear, but the issue was more of
        a clarification question than feedback \\ \hline
        \href{https://github.com/PlutosCapstone/Plutos/issues/44}{44} & Peer
        Review & peer-review[team 23]: Data retention and Deletion Policies  & $\checkmark$ &
        \href{https://github.com/PlutosCapstone/Plutos/pull/179}{179} & Changes
        made \\ \hline
        \href{https://github.com/PlutosCapstone/Plutos/issues/105}{105} & TA
        Feedback & docs(SRS): Include formal/math specs & $\checkmark$ & \href{https://github.com/PlutosCapstone/Plutos/pull/300}{300}
        &  Added formal math specifications in SRS for OCR model, data constraints, budget calculation and receipt/receipt item data definitions \\
        % USE THIS FOR COPYING:
        % \href{https://github.com/PlutosCapstone/Plutos/issues/x}{x} & Peer
        % Review &  &  & \href{https://github.com/PlutosCapstone/Plutos/pull/x}{x}
        % &   \\ 
    \bottomrule
    \end{tabular}%
    }
\end{table}


\begin{table}[H]
    \caption{Hazard Analysis Issues}
    \noindent 
    \makebox[\textwidth]{%
    \begin{tabular}{p{1.5cm} p{2cm} p{3.5cm} c c p{4cm}} 
    \toprule 
        \textbf{Issue Number} & \textbf{Source} & \textbf{Issue Title} &
        \textbf{Addressed?} & \textbf{PR} & \textbf{Comments} \\ 
        \midrule
        \href{https://github.com/PlutosCapstone/Plutos/issues/111}{111} & TA &
        docs(hazards): Add list of tables & $\checkmark$ &
        \href{https://github.com/PlutosCapstone/Plutos/pull/288}{288} &  Added
        table \\ \hline
        \href{https://github.com/PlutosCapstone/Plutos/issues/112}{112} & TA &
        docs(hazards): Put constants in constants section & $\checkmark$ &
        \href{https://github.com/PlutosCapstone/Plutos/pull/288}{288} &  Refered
        to symbolic constants \\ \hline
        \href{https://github.com/PlutosCapstone/Plutos/issues/113}{113} & TA &
        docs(hazards): Fix hazard recommended action & $\checkmark$ &
        \href{https://github.com/PlutosCapstone/Plutos/pull/299}{299} &  Changed
        recommended action \\ \hline
        \href{https://github.com/PlutosCapstone/Plutos/issues/51}{51} & Peer
        Review & Peer Review (hazards) - Expand Analysis on External System
        Interactions &  &
        \href{https://github.com/PlutosCapstone/Plutos/pull/x}{x} &   \\ 
        \href{https://github.com/PlutosCapstone/Plutos/issues/52}{52} & Peer
        Review & Peer Review (hazards) - Enhance Network Failure Handling and
        Data Integrity Measures &  &
        \href{https://github.com/PlutosCapstone/Plutos/pull/x}{x} &   \\ 
        \href{https://github.com/PlutosCapstone/Plutos/issues/53}{53} & Peer
        Review & Peer Review (hazards) - Improve Hazard Mitigation User Feedback &  & \href{https://github.com/PlutosCapstone/Plutos/pull/x}{x}
        &   \\ \hline
        \href{https://github.com/PlutosCapstone/Plutos/issues/54}{54} & Peer
        Review & Peer Review (hazards) - some assumptions needed for user end
        equipment & $\checkmark$ &
        \href{https://github.com/PlutosCapstone/Plutos/pull/187}{187} & Added
        suggested assumptions \\ \hline
        \href{https://github.com/PlutosCapstone/Plutos/issues/55}{55} & Peer
        Review & Peer Review (hazards) - concern about problem resolvement &
        $\checkmark$ &
        \href{https://github.com/PlutosCapstone/Plutos/pull/187}{187} & Added
        suggested changes \\ 
        % USE THIS FOR COPYING:
        % \href{https://github.com/PlutosCapstone/Plutos/issues/x}{x} & Peer
        % Review &  &  & \href{https://github.com/PlutosCapstone/Plutos/pull/x}{x}
        % &   \\ 
    \bottomrule
    \end{tabular}%
    }
\end{table}


\subsection{Design and Design Documentation}

\begin{table}[H]
    \caption{Design Doc Issues}
    \noindent 
    \makebox[\textwidth]{%
    \begin{tabular}{p{1.5cm} p{2cm} p{3.5cm} c c p{4cm}} 
    \toprule 
        \textbf{Issue Number} & \textbf{Source} & \textbf{Issue Title} &
        \textbf{Addressed?} & \textbf{PR} & \textbf{Comments} \\ 
        \midrule
        \href{https://github.com/PlutosCapstone/Plutos/issues/158}{158} & Peer Review &
        Module hierarchy diagram Link Issue & $\checkmark$ &
        \href{https://github.com/PlutosCapstone/Plutos/pull/302}{302} &  Adjusted
        DAG diagram \\ \hline
        \href{https://github.com/PlutosCapstone/Plutos/issues/239}{239} & TA Review &
        DetDesDoc semantics - add 'main' UI module & $\checkmark$ &
        \href{https://github.com/PlutosCapstone/Plutos/pull/302}{302} &  Adjusted
        DAG diagram and added Main UI module to module hierarchy and DAG \\ \hline
        \href{https://github.com/PlutosCapstone/Plutos/issues/238}{238} & TA Review &
        DetDesDoc semantics - add state diagram & $\checkmark$ &
        \href{https://github.com/PlutosCapstone/Plutos/pull/302}{302} &  Added
        state diagram to Figma \\ \hline
        \href{https://github.com/PlutosCapstone/Plutos/issues/240}{240} & TA Review &
        DetDesDoc semantics - exception handling specs & $\checkmark$ &
        \href{https://github.com/PlutosCapstone/Plutos/pull/302}{302} &  Added
        exception handling section for item misclassification \\ \hline
        \href{https://github.com/PlutosCapstone/Plutos/issues/159}{159} & Peer Review &
        OCR Processing Module Issue & $\checkmark$ &
        \href{https://github.com/PlutosCapstone/Plutos/pull/302}{302} &  Added
        exception handling section for OCR module \\
        % USE THIS FOR COPYING:
        % \href{https://github.com/PlutosCapstone/Plutos/issues/x}{x} & Peer
        % Review &  &  & \href{https://github.com/PlutosCapstone/Plutos/pull/x}{x}
        % &   \\ 
    \bottomrule
    \end{tabular}%
    }
\end{table}

\subsection{VnV Plan and Report}

\begin{table}[H]
    \caption{VnV Plan Issues}
    \noindent 
    \makebox[\textwidth]{%
    \begin{tabular}{p{1.5cm} p{2cm} p{3.5cm} c c p{4cm}} 
    \toprule 
        \textbf{Issue Number} & \textbf{Source} & \textbf{Issue Title} &
        \textbf{Addressed?} & \textbf{PR} & \textbf{Comments} \\ 
        \midrule
        \href{https://github.com/PlutosCapstone/Plutos/issues/74}{74} & Peer
        Review & Automated Testing and Verification Tools & $\times$ & &  Daily
        sanity checks aren't necessary as we are running the tests and linter
        pipelines on every PR. This will ensure that the app/features work
        seamlessely after every change. \\ 
        \hline
        \href{https://github.com/PlutosCapstone/Plutos/issues/75}{75} & Peer
        Review & Load Testing for Concurrent Users & $\times$ &
        \href{https://github.com/PlutosCapstone/Plutos/pull/298}{298} &  Already
        addressed in initial version of VnVPlan `Performance tests can be
        conducted to measure the app's speed and reliability, especially when
        processing large receipts or handling multiple users'. \\ 
        \hline
        \href{https://github.com/PlutosCapstone/Plutos/issues/120}{120} & TA &
        docs(vnv): system tests for FRs & $\checkmark$ &
        \href{https://github.com/PlutosCapstone/Plutos/pull/290}{290} &  Adjust
        control of system tests. \\ 
        \hline
        \href{https://github.com/PlutosCapstone/Plutos/issues/119}{119} & TA &
        docs(vnv): add to testing plan & $\checkmark$ &
        \href{https://github.com/PlutosCapstone/Plutos/pull/290}{290} &  Adjust
        functional testing criteria. \\
        \hline
        \href{https://github.com/PlutosCapstone/Plutos/issues/52}{52} & Peer Review & Enhance Network Failure Handling and Data Integrity Measures & $\checkmark$ &
        \href{https://github.com/PlutosCapstone/Plutos/pull/185}{185} &  \\
    \bottomrule
    \end{tabular}%
    }
\end{table}


\section{Challenge Level and Extras}

\subsection{Challenge Level}

\plt{State the challenge level (advanced, general, basic) for your project.  Your challenge level should exactly match what is included in your problem statement.  This should be the challenge level agreed on between you and the course instructor.}

\subsection{Extras}

\plt{Summarize the extras (if any) that were tackled by this project.  Extras
can include usability testing, code walkthroughs, user documentation, formal
proof, GenderMag personas, Design Thinking, etc.  Extras should have already
been approved by the course instructor as included in your problem statement.}

\section{Design Iteration (LO11 (PrototypeIterate))}

\plt{Explain how you arrived at your final design and implementation.  How did
the design evolve from the first version to the final version?} 

\plt{Don't just say what you changed, say why you changed it.  The needs of the
client should be part of the explanation.  For example, if you made changes in
response to usability testing, explain what the testing found and what changes
it led to.}

\section{Design Decisions (LO12)}

\plt{Reflect and justify your design decisions.  How did limitations,
 assumptions, and constraints influence your decisions?  Discuss each of these
 separately.}

\section{Economic Considerations (LO23)}

\plt{Is there a market for your product? What would be involved in marketing your 
product? What is your estimate of the cost to produce a version that you could 
sell?  What would you charge for your product?  How many units would you have to 
sell to make money? If your product isn't something that would be sold, like an 
open source project, how would you go about attracting users?  How many potential 
users currently exist?}

\section{Reflection on Project Management (LO24)}

\plt{This question focuses on processes and tools used for project management.}

\subsection{How Does Your Project Management Compare to Your Development Plan}

\plt{Did you follow your Development plan, with respect to the team meeting plan, 
team communication plan, team member roles and workflow plan.  Did you use the 
technology you planned on using?}

\subsection{What Went Well?}

\plt{What went well for your project management in terms of processes and 
technology?}

\subsection{What Went Wrong?}

\plt{What went wrong in terms of processes and technology?}

\subsection{What Would you Do Differently Next Time?}

\plt{What will you do differently for your next project?}

\section{Reflection on Capstone}

\plt{This question focuses on what you learned during the course of the capstone project.}

\subsection{Which Courses Were Relevant}

\plt{Which of the courses you have taken were relevant for the capstone project?}

\subsection{Knowledge/Skills Outside of Courses}

\plt{What skills/knowledge did you need to acquire for your capstone project
that was outside of the courses you took?}

\end{document}
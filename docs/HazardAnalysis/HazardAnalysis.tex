\documentclass{article}

\usepackage{booktabs}
\usepackage{tabularx}
\usepackage{hyperref}
\usepackage[hmargin=3cm]{geometry}
\usepackage{pdflscape}
\usepackage{longtable}
\usepackage{enumitem}
\usepackage{lscape}


\hypersetup{
    colorlinks=true,       % false: boxed links; true: colored links
    linkcolor=red,          % color of internal links (change box color with linkbordercolor)
    citecolor=green,        % color of links to bibliography
    filecolor=magenta,      % color of file links
    urlcolor=cyan           % color of external links
}

\title{Hazard Analysis\\\progname}

\author{\authname}

\date{}

%% Comments

\usepackage{color}

\newif\ifcomments\commentstrue %displays comments
%\newif\ifcomments\commentsfalse %so that comments do not display

\ifcomments
\newcommand{\authornote}[3]{\textcolor{#1}{[#3 ---#2]}}
\newcommand{\todo}[1]{\textcolor{red}{[TODO: #1]}}
\else
\newcommand{\authornote}[3]{}
\newcommand{\todo}[1]{}
\fi

\newcommand{\wss}[1]{\authornote{blue}{SS}{#1}} 
\newcommand{\plt}[1]{\authornote{magenta}{TPLT}{#1}} %For explanation of the template
\newcommand{\an}[1]{\authornote{cyan}{Author}{#1}}

%% Common Parts

\newcommand{\progname}{ProgName} % PUT YOUR PROGRAM NAME HERE
\newcommand{\authname}{Team \#, Team Name
\\ Student 1 name
\\ Student 2 name
\\ Student 3 name
\\ Student 4 name} % AUTHOR NAMES                  

\usepackage{hyperref}
    \hypersetup{colorlinks=true, linkcolor=blue, citecolor=blue, filecolor=blue,
                urlcolor=blue, unicode=false}
    \urlstyle{same}
                                


\begin{document}

\maketitle
\thispagestyle{empty}

~\newpage

\pagenumbering{roman}

\begin{table}[hp]
\caption{Revision History} \label{TblRevisionHistory}
\begin{tabularx}{\textwidth}{llX}
\toprule
\textbf{Date} & \textbf{Developer(s)} & \textbf{Change}\\
\midrule
10/23/2024 & Angela & Initial draft\\
10/25/2024 & Jason & Table for section 5 \\
10/25/2024 & Payton, Eric, Fondson, Jason, Angela & Final Rev 0 \\
10/28/2024 & Eric & Update section 3\\
01/29/2025 & Payton & Revising sections 3 \& 5\\
01/29/2025 & Payton & Revising sections 3 \& 5\\
... & ... & ... \\
\bottomrule
\end{tabularx}
\end{table}

~\newpage

\tableofcontents

~\newpage

\pagenumbering{arabic}

\section{Introduction}

A hazard in the context of this document is any property or condition that may
lead to harm or damage to the Plutos system or its users. Potential losses due
to these hazards may include loss of application functionality, performance, or
accuracy, or breaches of user privacy or data. The following sections will
identify hazards within the system and discuss the controls in place for their
mitigation.


\section{Scope and Purpose of Hazard Analysis}

This document aims to provide a comprehensive hazard analysis of the Plutos
system. It identifies hazards within the system, outlines measures to mitigate
them, and specifies the safety and security requirements derived from this
analysis. The analysis will follow the Failure Mode and Effect Analysis (FMEA)
approach. The analysis aims to discover the potential failure modes within the
system and develop a mitigation plan to reduce the risk of failure. 


\section{System Boundaries and Components}

The system will be divided into the following components:
\begin{enumerate}
    \item The Plutos application, which consists of:
    \begin{enumerate}
        \item \textbf{The database}: The database is where the user's receipts
        and profile data will be stored.
		\begin{itemize}
			\item Hazards:
				\begin{enumerate}
					\item SQL Injection - Attackers can manipulate SQL queries through user inputs, potentially accessing or altering the database.
				\end{enumerate}
			\item Mitigation:
				\begin{enumerate}
					\item Use parameterized queries or prepared statements to separate SQL code from data inputs, and validate and sanitize all user inputs.
				\end{enumerate}
		\end{itemize}
        \item \textbf{The backend server}: The backend server is responsible for
        handling and serving requests from the client. It will interact with all
        the other components listed here.
		\begin{itemize}
			\item Hazards:
				\begin{enumerate}
					\item Lack of Rate Limiting - Excessive requests from users or automated scripts can overload the app and lead to denial of service.
					\item Dependency Vulnerabilities - Using third-party libraries and frameworks can introduce vulnerabilities if they are not regularly updated.
					\item Inadequate Error Handling - Poor error handling can expose stack traces or sensitive information to users, aiding attackers.
					\item Insecure API Endpoints - Unprotected or poorly secured API endpoints can expose sensitive data or allow unauthorized actions.
				\end{enumerate}
			\item Mitigation:
				\begin{enumerate}
					\item Implement rate limiting to control the number of requests a user can make in a given time frame and deploy web application firewalls (WAFs) to filter malicious traffic. 
					\item Regularly review and update dependencies, and use tools to scan for known vulnerabilities in third-party packages.
					\item Implement generic error messages for users and log detailed errors securely for developers.
					\item Implement proper authentication and authorization mechanisms, and use HTTPS to secure API communications.
				\end{enumerate}
		\end{itemize}
        \item \textbf{The frontend/user interface}: The frontend/user interface
        is responsible for displaying the appropriate views to the user and
        handling user interactions.
		\begin{itemize}
			\item Hazards:
				\begin{enumerate}
					\item Authentication Vulnerabilities - Weak authentication mechanisms can allow unauthorized users to access sensitive features of the app.
					\item Cross-Site Scripting (XSS) Attacks - Attackers may inject malicious scripts via images into the app, which can steal user data or compromise the app's functionality.
					\item Poor Session Management - Users may remain logged in indefinitely, increasing the risk of unauthorized access on shared or public devices.
					\item Client-Side Storage Vulnerabilities - Sensitive data stored in local storage or cookies can be easily accessed or manipulated by malicious scripts.
					\item Unprotected Routes - Sensitive pages may not have adequate access controls, allowing unauthorized users to access them.
				\end{enumerate}
			\item Mitigation:
				\begin{enumerate}
					\item Implement an IAM service (OAuth 2.0) that handles authentication and authorization amongst login sessions.
					\item Sanitize all user inputs and outputs to ensure that any HTML or JavaScript code is escaped or neutralized.
					\item Implement session expiration policies and automatically log out inactive users and regularly rotate session tokens to minimize the risk of token theft.
					\item Avoid storing sensitive information in client-side storage and use secure cookies for any necessary data.
					\item Implement role-based access control (RBAC) and ensure all routes are properly protected by authentication checks.
				\end{enumerate}
		\end{itemize}
        \item \textbf{The machine learning (ML) model}: The ML model is
        responsible for parsing and categorizing items from a picture of an
        itemized receipt.
    \end{enumerate}
    \item The user’s mobile device and camera setup
    \item Several links to external systems:
    \begin{enumerate}
        \item Plutos will interact with various external APIs and services, which can 
        impact its reliability and security. 
		\begin{itemize}
			\item Hazards:
				\begin{enumerate}
					\item Data exchange vulnerabilities with third-party services.
                    \item Dependency on external service availability, which may lead to system downtime.
                    \item Inadequate authentication mechanisms for external API calls.
				\end{enumerate}
			\item Mitigation:
				\begin{enumerate}
					\item Implement robust authentication and authorization protocols for all external API interactions.
                    \item Regularly audit and monitor third-party services for security compliance and performance reliability.
                    \item Establish fallback mechanisms to handle failures in external services, ensuring minimal disruption 
                    to the Plutos system.
                \end{enumerate}
        \end{itemize}
    \end{enumerate}
    \item Several links to external systems:
    \begin{enumerate}
        \item Plutos will interact with various external APIs and services, which can 
        impact its reliability and security. 
		\begin{itemize}
			\item Hazards:
				\begin{enumerate}
					\item Data exchange vulnerabilities with third-party services.
                    \item Dependency on external service availability, which may lead to system downtime.
                    \item Inadequate authentication mechanisms for external API calls.
				\end{enumerate}
			\item Mitigation:
				\begin{enumerate}
					\item Implement robust authentication and authorization protocols for all external API interactions.
                    \item Regularly audit and monitor third-party services for security compliance and performance reliability.
                    \item Establish fallback mechanisms to handle failures in external services, ensuring minimal disruption 
                    to the Plutos system.
                \end{enumerate}
        \end{itemize}
    \end{enumerate}
    \item Several links to external systems:
    \begin{enumerate}
        \item Plutos will interact with various external APIs and services, which can 
        impact its reliability and security. 
		\begin{itemize}
			\item Hazards:
				\begin{enumerate}
					\item Data exchange vulnerabilities with third-party services.
                    \item Dependency on external service availability, which may lead to system downtime.
                    \item Inadequate authentication mechanisms for external API calls.
				\end{enumerate}
			\item Mitigation:
				\begin{enumerate}
					\item Implement robust authentication and authorization protocols for all external API interactions.
                    \item Regularly audit and monitor third-party services for security compliance and performance reliability.
                    \item Establish fallback mechanisms to handle failures in external services, ensuring minimal disruption 
                    to the Plutos system.
                \end{enumerate}
        \end{itemize}
    \end{enumerate}
\end{enumerate}


\section{Critical Assumptions}

The project will be making the following critical assumptions: 
\begin{enumerate}
    \item The users will be
    using a mobile device running an up-to-date version of iOS or Android.
    \item Users are
    not expected to repeatedly input invalid images into the system (i.e., images
    that do not contain a receipt). While it is anticipated that users may
    occasionally submit an invalid image, it is assumed to not be a significant
    concern.
    \item The project assumes that users' cameras will function properly and produce 
    images of adequate quality for analysis.
    \item Issues such as low-quality images or invalid pictures may arise from the 
    user's equipment (e.g., a malfunctioning camera) and are considered outside the 
    project's scope.
\end{enumerate}  


\newgeometry{bottom=25mm,hmargin=1.5cm,vmargin=0.5cm, landscape}
\begin{landscape}

\section{Failure Mode and Effect Analysis}

\begin{longtable}{|p{3cm}|p{4cm}|p{5cm}|p{5cm}|p{5cm}|p{1cm}|p{0.75cm}|}
    \caption{Failure Mode and Effect Analysis Table} \label{tab:long} \\
    
    \hline
    \multicolumn{1}{|c|}{\textbf{Design Function}} & 
    \multicolumn{1}{c|}{\textbf{Failure Modes}} & 
    \multicolumn{1}{c|}{\textbf{Effects of Failure}} & 
    \multicolumn{1}{c|}{\textbf{Causes of Failure}} & 
    \multicolumn{1}{c|}{\textbf{Recommended Action}} & 
    \multicolumn{1}{c|}{\textbf{SR}} & 
    \multicolumn{1}{c|}{\textbf{Ref}} \\
    \hline
    \endfirsthead
    
    \hline
    \multicolumn{7}{|c|}{\textit{Continued from previous page}} \\
    \hline
    \multicolumn{1}{|c|}{\textbf{Design Function}} & 
    \multicolumn{1}{c|}{\textbf{Failure Modes}} & 
    \multicolumn{1}{c|}{\textbf{Effects of Failure}} & 
    \multicolumn{1}{c|}{\textbf{Causes of Failure}} & 
    \multicolumn{1}{c|}{\textbf{Recommended Action}} & 
    \multicolumn{1}{c|}{\textbf{SR}} & 
    \multicolumn{1}{c|}{\textbf{Ref}} \\
    \hline
    \endhead
    
    \hline
    \endfoot
    
    Authenticate user & 
    Non-human accounts exist on the server & 
    Increased traffic to the server, invalid inputs, performance deterioration & 
    A bot account logs into the application & 
    Ensure account creation includes a captcha & 
    FR3, SR1 & 
    H1-1 \\
    \cline{2-7}
     & 
    User account information has been compromised due to unauthorized access & 
    User can no longer access their account and their data might be compromised or lost & 
    Unauthorized account access & 
    Enforce strong user passwords, allow users to reset their password & 
    FR6, SR7, SR12 & 
    H1-2 \\
    \hline
    % row 2
    Accept image & 
    Camera does not open &
    The user cannot input an image to be processed by the system. &
    \begin{enumerate}[label=(\alph*), leftmargin=0.5cm]
        \item Application does not have access to user's camera
        \item The application is bugged out
    \end{enumerate} &
    \begin{enumerate}[label=(\alph*), leftmargin=0.5cm]
        \item Prompt the user to allow camera access for the application
        \item The user should restart the application. 
    \end{enumerate} &
    FR7, NFR13, SR2 &
    H2-1\\
    \cline{2-7}
    % row 3
     & 
    Photo library doesn't open &
    Same as H2-1 &
    Same as H2-1 &
    Same as H2-1 &
    FR8, NFR13, SR3 &
    H2-2\\
    \hline
    Process receipt image &
    Image is not processable &
    Poor input into machine learning model; may cause incorrect results or cause model to run for a long time &
    \begin{enumerate}[label=(\alph*), leftmargin=0.5cm]
        \item Poor quality receipt (e.g., paper is crumpled or wrinkled, light ink) 
        \item Poor photo quality (e.g., blurry, part of receipt cut out from frame, low lighting)
        \item The image does not contain a receipt
    \end{enumerate} &
    \begin{enumerate}[label=(\alph*), leftmargin=0.5cm]
        \item Prompt the user to retry due to poor photo quality and make sure that the input is actually a receipt
        \item Same as H3-1a
        \item Run image validation before passing the image through the ML model (e.g., check that there is a paper with text in the image)
    \end{enumerate} &
    FR16, NFR13, SR4, SR6 &
    H3-1\\ 
    \cline{2-7}
    &
    Optical character recognition (OCR) failing or incorrectly parsing text &
    Incorrectly parsed text, missing items, incorrect categorization or costs &
    \begin{enumerate}[label=(\alph*), leftmargin=0.5cm]
        \item Receipt uses different font sizes or styles
        \item Receipt quality causes some letters to be difficult to recognize (e.g., 0 and O, T and I)
    \end{enumerate} &
    \begin{enumerate}[label=(\alph*), leftmargin=0.5cm]
        \item Inform user that the OCR isn’t working after 3 failed attempts and ask user to manually input receipt expenses
    \end{enumerate} &
    FR16, NFR4 &
    H3-2\\
    \cline{2-7}
    &
    Incorrect item categorization &
    Items are miscategorized, resulting in incorrect insights &
    \begin{enumerate}[label=(\alph*), leftmargin=0.5cm]
        \item OCR incorrectly parses text or fails to read some text
        \item Ambiguous item name
        \item Item has not been identified in the past
    \end{enumerate} &
    \begin{enumerate}[label=(\alph*), leftmargin=0.5cm]
        \item Refer to H3-2.
        \item Prompt the user to confirm the category of this object and give it an alias for helping in future recognition.
        \item See H3-3b
    \end{enumerate} &
    FR17, NFR2 &
    H3-3\\
    \cline{2-7}
    & 
    Processing image takes more than X seconds to complete &
    \begin{enumerate}[label=(\alph*), leftmargin=0.5cm]
        \item Lower user satisfaction
    \end{enumerate} &
    \begin{enumerate}[label=(\alph*), leftmargin=0.5cm]
        \item User’s device is low on memory or has other tasks running in the background
        \item Large image size/high image resolution
        \item OCR model not optimized (especially for that specific use case)
        \item Poor pre-processing  
    \end{enumerate} &
    \begin{enumerate}[label=(\alph*), leftmargin=0.5cm]
        \item Inform the user that the task is taking longer than expected and show troubleshooting tips.
        \item See H3-2
        \item See H3-2
        \item See H3-2
    \end{enumerate} &
    NFR8, NFR16 &
    H3-4\\
    \hline
    Save new receipt input (image details and model results) &
    Cannot connect to database &
    Receipt input cannot be saved after being processed. &
    \begin{enumerate}[label=(\alph*), leftmargin=0.5cm]
        \item Poor network connection
        \item Database server downtime
    \end{enumerate} &
    \begin{enumerate}[label=(\alph*), leftmargin=0.5cm]
        \item Prompt the user to check their network connection
        \item Inform the user of the server error; system will try again at a later time (data will be stored locally and then backup up)
    \end{enumerate} &
    FR15, NFR13, SR5, SR8 &
    H4-1 \\
    \hline
    The system suggests user budget and goals &
    The system suggests a budget that is lower than the user's total monthly expenses (i.e. unattainable budget) &
    \begin{enumerate}[label=(\alph*), leftmargin=0.5cm]
        \item Lower user satisfaction
        \item User is unable to budget correspondingly to the suggested system budget 
    \end{enumerate} &
    \begin{enumerate}[label=(\alph*), leftmargin=0.5cm]
        \item Algorithm to calculate the budget fails 
    \end{enumerate} &
    \begin{enumerate}[label=(\alph*), leftmargin=0.5cm]
        \item Provide an option to the user to recalculate monthly budget with a given minimum monthly expense (user enters minimum monthly expense and system recalculates budget) 
    \end{enumerate} &
    NFR5 &
    H5-1\\
    \hline
    Data Synchronization &
    Network failure during data transmission &
    \begin{enumerate}[label=(\alph*), leftmargin=0.5cm]
        \item Loss of data integrity
        \item Inconsistent user experience
        \item Potential data loss
    \end{enumerate} &
    \begin{enumerate}[label=(\alph*), leftmargin=0.5cm]
        \item Poor network connectivity
        \item Server downtime
    \end{enumerate} &
    \begin{enumerate}[label=(\alph*), leftmargin=0.5cm]
        \item Implement data redundancy measures to ensure data is not lost during transmission.
        \item Establish real-time data backup protocols to maintain data integrity.
        \item Develop detailed protocols for data validation and synchronization after network recovery to ensure consistency.
    \end{enumerate} &
    NFR20 &
    H6-1 \\
    \hline
    Image Processing Errors &
    Repeated image rejection due to poor quality or prolonged processing times &
    \begin{enumerate}[label=(\alph*), leftmargin=0.5cm]
        \item User frustration leading to potential abandonment of the application
        \item Decreased user satisfaction and retention
    \end{enumerate} &
    \begin{enumerate}[label=(\alph*), leftmargin=0.5cm]
        \item Poor quality of input images (e.g., blurry, low lighting)
        \item High processing times due to large image sizes or device limitations
    \end{enumerate} &
    \begin{enumerate}[label=(\alph*), leftmargin=0.5cm]
        \item Provide clear and specific error messages indicating the issue with the image quality.
        \item Implement visual cues and real-time guidance to assist users in troubleshooting image quality issues.
        \item Introduce a retry mechanism with tips for improving image capture (e.g., lighting, focus).
    \end{enumerate} &
    NFR16, NFR17 &
    H7-1 \\
\end{longtable}
    \newpage{}
\end{landscape}
\restoregeometry

\section{Safety and Security Requirements}

These are newly identified requirements that will be added to the Software
Requirements Specification (SRS) document. They will be relabelled and added to
appropriate functional and non-functional requirements sections in the SRS.

\begin{enumerate}[label=SR\arabic*]
    \item Account creation must include a captcha to verify that the account is
    being created by a human.\\Rationale: Bot accounts can cause increased
    traffic to the server, invalid inputs, and performance deterioration.
    \\Associated Hazards: H1-1
    \item The application must prompt the user to allow camera access upon
    needing to open the camera. If camera access is disabled, the application
    must notify the user and show instructions on how to enable camera
    access.\\Rationale: If the camera does not open, the user cannot input an
    image to be processed by the system.\\Associated Hazards: H2-1
    \item The application must prompt the user to allow photo library access upon
    needing to open the photo library. If photo library access is disabled, the
    application must notify the user and show instructions on how to enable
    photo library access.\\Rationale: If the photo library does not open, the
    user cannot input an image to be processed by the system.\\Associated
    Hazards: H2-2
    \item The application will run image validation before passing the image
    through the ML model; the validation will check that there is a paper with
    text in the image.\\Rationale: If the image is not processable, it may cause
    incorrect results or cause the model to run for a long time.\\Associated
    Hazards: H3-1
    \item The database will be store a local copy of receipts within the app
    until a connection to the server can be established.\\Rationale: If the
    system cannot connect to the database, receipt input cannot be saved after
    being processed.\\Associated Hazards: H4-1 
    \item The application will apply image sanitization techniques to ensure
    that malicious content cannot be injected into the app via uploaded images
    (e.g. XSS). \\Rationale: Protecting against malicious content injection
    will help prevent attacks that could compromise user data or the system.
    \\Associated Hazards: H3-1
	\item User account passwords must be a combination of uppercase letters,
	lowercase letters, numbers, and symbols, and be at least 8 characters
	long.\\Rationale: Weak passwords can lead to unauthorized access to user
	accounts.\\Associated Hazards: H1-2
    \item The application will ensure that images of receipts are securely stored
    using access controls and encryption and should be deleted when they are no
    longer needed or upon user request. \\Rationale: Protecting receipt images
    will help prevent unauthorized access to sensitive financial data.\\Associated
    Hazards: N/A
	\item The application will be regularly updated to address newly discovered
	vulnerabilities, both in the application code along with any third-party
	libraries of frameworks. This will be done on an iterative basis with
	periodic patch releases (e.g. patch release every 3 weeks). \\Rationale:
	Regular security updates will help protect against newly discovered
	vulnerabilities that could be exploited by attackers.\\Associated Hazards:
	N/A
	\item Before storing or processing any receipts that are uploaded, all
	sensitive information such as account numbers or personal details will be
	anonymized to protect user privacy. \\Rationale: Anonymizing data will help
	protect user privacy and prevent unauthorized access to sensitive data. This
	is especially important when using data for AI training or analysis.
	\\Associated Hazards: N/A
	\item While the assumption is that users will be on a mobile device, the
	device must be secure (e.g. detecting if the device is jailbroken or
	rooted). \\Rationale: This can help against local data
	theft or tampering. \\Associated Hazards: N/A
    \item The application will allow users to remotely wipe sensitive financial
	data and receipts from their devices in the case of loss or theft.
	\\Rationale: This will help protect user data in the event of a lost or
    stolen device.  \\Associated Hazards: H1-2
\end{enumerate}

\section{Roadmap}

Due to the time constraint of the capstone timeline, not all safety/security
requirements identified in Section 6 of this hazards analysis document will be
implemented; only a subset will be on the roadmap.\\

\noindent The requirements that will be implemented as part of the capstone time
will be SR1, SR2, SR3, and SR7. The rest of the requirements will be implemented
in the future.\\

As noted in the Section 9 of the \href{https://github.com/PlutosCapstone/Plutos/blob/main/docs/SRS/SRS.pdf}{SRS document}, there are four phases for the
development plan. The new requirements will be implemented in the following phases:
\begin{enumerate}
    \item Phase 1: SR1, SR7
    \item Phase 2: SR2, SR3
\end{enumerate}



\newpage{}

\section*{Appendix --- Reflection}

The purpose of reflection questions is to give you a chance to assess your own
learning and that of your group as a whole, and to find ways to improve in the
future. Reflection is an important part of the learning process.  Reflection is
also an essential component of a successful software development process.  

Reflections are most interesting and useful when they're honest, even if the
stories they tell are imperfect. You will be marked based on your depth of
thought and analysis, and not based on the content of the reflections
themselves. Thus, for full marks we encourage you to answer openly and honestly
and to avoid simply writing ``what you think the evaluator wants to hear.''

Please answer the following questions.  Some questions can be answered on the
team level, but where appropriate, each team member should write their own
response:

\begin{enumerate}

\item What went well while writing this deliverable?

The overall process while writing this deliverable was smooth and efficient as
we were quickly able to identify the potential hazards related to our project.
We brainstormed several ambiguous sections or things we thought were a bit
unclear within this analysis document, and were able to get very clear answers
from our helpful TA. The team worked well together as we all put in our best
efforts and supported one another when completing this task. 

Using the FMEA (Failure Mode and Effect Analysis) approach helped streamline the
hazard identification process. Breaking down the system into components allowed
for a clear understanding of where risks might occur. Writing the deliverable
helped the team clarify and solidify their understanding of how the receipt
scanner, the AI model, and other system components interact, making it easier to
identify hazards.

\item What pain points did you experience during this deliverable, and how did
you resolve them?

At first, it was challenging to define the potential failure modes, especially
for components like the machine learning model. The team resolved this by
conducting additional research on common failure points in similar systems and
reviewing how AI models typically behave with poor input data. Another challenge
was balancing realistic assumptions about user behavior with potential risks.
For example, while assuming users won’t repeatedly input invalid images, we
acknowledged this could still happen. We resolved this by planning mitigation
strategies for those edge cases.

\item Which of your listed risks had your team thought of before this
deliverable, and which did you think of while doing this deliverable? For the
latter ones (ones you thought of while doing the Hazard Analysis), how did they
come about?

We had already considered risks related to image quality (e.g., blurry or
incomplete receipt images) and network connectivity issues (e.g., users not
being able to connect to the database).

While working on the hazard analysis, we realized potential risks like Optical
Character Recognition (OCR) misinterpretation due to varied receipt fonts and ML
model processing time under different device conditions (e.g., low memory or
poor network). These came up while brainstorming as a team and thinking about
the specific steps in image processing and how the system handles diverse input.

\item Other than the risk of physical harm, list at least 2 other types of risk
in software products. Why are they important to consider?

Two other risks that are apparent in software products are security and
reliability risks. 

Security vulnerabilities can lead to issues such as data breaches, unauthorized
access or identity theft, as well as collateral damages, whether it be financial
losses or reputational damage. This is considered a risk and is important to
consider as it creates an opportunity for malicious users to exploit weaknesses
in software systems, which can have a range of detrimental consequences.
Examples include operational disruptions, intellectual property theft,
ransomware attacks, etc.

As for reliability, it is mostly concerned with when software fails to function
consistently, such as having frequent downtimes. This can affect the user’s
experience, leading to a loss of productivity or customer dissatisfaction. Both
of these can lead to potential revenue loss. This is classified as a risk and is
important to consider because unreliable software can lead to negative
consequences, which affect not only the users but also the organization that
provides the software. The damages can be both monetary and non-monetary, such
as losing user trust/loyalty, reputational damage, and associated compliance and
legal risks.
\end{enumerate}

\end{document}
\documentclass{article}

\usepackage{tabularx}
\usepackage{booktabs}

\title{Problem Statement and Goals\\\progname}

\author{\authname}

\date{}

%% Comments

\usepackage{color}

\newif\ifcomments\commentstrue %displays comments
%\newif\ifcomments\commentsfalse %so that comments do not display

\ifcomments
\newcommand{\authornote}[3]{\textcolor{#1}{[#3 ---#2]}}
\newcommand{\todo}[1]{\textcolor{red}{[TODO: #1]}}
\else
\newcommand{\authornote}[3]{}
\newcommand{\todo}[1]{}
\fi

\newcommand{\wss}[1]{\authornote{blue}{SS}{#1}} 
\newcommand{\plt}[1]{\authornote{magenta}{TPLT}{#1}} %For explanation of the template
\newcommand{\an}[1]{\authornote{cyan}{Author}{#1}}

%% Common Parts

\newcommand{\progname}{ProgName} % PUT YOUR PROGRAM NAME HERE
\newcommand{\authname}{Team \#, Team Name
\\ Student 1 name
\\ Student 2 name
\\ Student 3 name
\\ Student 4 name} % AUTHOR NAMES                  

\usepackage{hyperref}
    \hypersetup{colorlinks=true, linkcolor=blue, citecolor=blue, filecolor=blue,
                urlcolor=blue, unicode=false}
    \urlstyle{same}
                                


\begin{document}

\maketitle

\begin{table}[hp]
\caption{Revision History} \label{TblRevisionHistory}
\begin{tabularx}{\textwidth}{llX}
\toprule
\textbf{Date} & \textbf{Developer(s)} & \textbf{Change}\\
\midrule
09/18/2024 & Angela Wang & Initial Draft\\
09/20/2024 & Jason Tan & Environment section and touch ups\\
09/24/2024 & Eric Chen & Update Appendix Reflection Questions\\
... & ... & ...\\
\bottomrule
\end{tabularx}
\end{table}

\section{Problem Statement}

% \wss{You should check your problem statement with the
% \href{https://github.com/smiths/capTemplate/blob/main/docs/Checklists/ProbState-Checklist.pdf}
% {problem statement checklist}.} 

% \wss{You can change the section headings, as long as you include the required
% information.}

\subsection{Problem}

Young adults often face challenges in managing their finances effectively,
especially when it comes to tracking expenses and budgeting. Despite
advancements in artificial intelligence (AI) and automation, many budgeting apps
still require manual data entry or calculations, resulting in an inefficient and
potentially inaccurate process. This inconvenience often leads users to poorly
manage their budgeting or abandon it altogether, hindering their ability to
optimize spending habits and achieve their financial goals.


\subsection{Inputs and Outputs}

% \wss{Characterize the problem in terms of ``high level'' inputs and outputs.  
% Use abstraction so that you can avoid details.}

\textbf{Inputs:} User's receipt photos and desired bugeting goals.\\
\textbf{Outputs:} Visualizations of the user's spending allocations in
comparison to their set budget, and recommendations for how they may adjust
their spending to meet their goals.

\subsection{Stakeholders}

Stakeholders include anyone who is looking to better manage their finances, set
budget goals, and track their spending, with a focus on first and second-year
university students who are just starting to live on their own.

\subsection{Environment}

% \wss{Hardware and software environment}

The software product will be compatible for Android and iOS mobile devices with
a functional camera.

\section{Goals}

Our goals include the following:
\begin{itemize}
    \item Develop a machine learning model that can accurately
    (\textgreater90\%) parse commonly purchased items (e.g., groceries, cleaning
    supplies) from a picture of a receipt.
    \item Develop a machine learning model that can accurately
    (\textgreater90\%) categorize items into approprite, pre-defined spending
    categories. 
    \item Develop a mobile application that allows users to take a picture of
    their receipt or manually input their expenses. These expenses would be
    stored in a database so that the user can review their spending history.
    \item Within the application, display visualizations of the user's purchases
    and spending allocations, and provide recommendations for how they may
    adjust their spending to meet their budget goals. These should be catered to
    the user's personal spending habits and goals.
    \item Allow users to set budget goals over different time intervals (e.g.,
    short-term, long-term) and track their progress towards these goals.
\end{itemize}

\section{Stretch Goals}
\begin{itemize}
    \item Build upon the base machine learning model to train on the user's
    personal spending data to provide more accurate item parsing and
    categorization (i.e., for items with similar names, the model can learn
    which category the user typically assigns them to).
    \item Build upon the base machine learning model to categorize items using
    user-defined/customizable spending categories.
    \item Build upon the base machine learning model to predict future spending
    based on the user's spending history and provide recommendations for how
    they can adjust their spending to meet their budget goals.
    \item Gamify the application to make it more engaging and encourate users to
    meet their budget goals and develop better spending habits.
    \item Allow users to input expenses through speech recognition, where the
    application can parse the user's speech and categorize the items
    accordingly.
\end{itemize}

\section{Challenge Level and Extras}

% \wss{State your expected challenge level (advanced, general or basic).  The
% challenge can come through the required domain knowledge, the implementation
% or something else.  Usually the greater the novelty of a project the greater
% its challenge level.  You should include your rationale for the selected
% level. Approval of the level will be part of the discussion with the
% instructor for approving the project.  The challenge level, with the approval
% (or request) of the instructor, can be modified over the course of the term.}

% \wss{Teams may wish to include extras as either potential bonus grades, or to
% make up for a less advanced challenge level.  Potential extras include
% usability testing, code walkthroughs, user documentation, formal proof,
% GenderMag personas, Design Thinking, etc.  Normally the maximum number of
% extras will be two.  Approval of the extras will be part of the discussion
% with the instructor for approving the project.  The extras, with the approval
% (or request) of the instructor, can be modified over the course of the term.}

The expected challenge level is \textbf{general}. The primary challenge of the
project is developing a machine learning model that can accurately parse items
from a picture of a receipt, and to categorize them into appropriate spending
categories. This requires a strong understanding of training and tuning models
on image data to achieve high accuracy. Additionally, different items across
various stores may have similar names or be difficult to recognize, which adds
to the complexity of the task. The other part of the project is to develop a
user-friendly mobile application, which is a more general software engineering
component.

Furthermore, our team plans to include the following two extras:
\begin{itemize}
    \item \textbf{Requirements elicitation}: We will conduct interviews and a
    survey to gather requirements from potential users to determine user needs
    and preferences.
    \item \textbf{Usability testing}: We will ask potential users to test the
    application and provide feedback on its usability and functionality.

\end{itemize}

\newpage{}

\section{Appendix --- Reflection}

The purpose of reflection questions is to give you a chance to assess your own
learning and that of your group as a whole, and to find ways to improve in the
future. Reflection is an important part of the learning process.  Reflection is
also an essential component of a successful software development process.  

Reflections are most interesting and useful when they're honest, even if the
stories they tell are imperfect. You will be marked based on your depth of
thought and analysis, and not based on the content of the reflections
themselves. Thus, for full marks we encourage you to answer openly and honestly
and to avoid simply writing ``what you think the evaluator wants to hear.''

Please answer the following questions.  Some questions can be answered on the
team level, but where appropriate, each team member should write their own
response:

\section{Appendix --- Team Charter}

\subsection{External Goals}

The team's objectives are to gain proficiency in AI/ML and to develop a well functioning application to showcase (and for people to try out) at the Capstone Expo. We also aim for the code and application to be clean and presentable for potential interviews.

\subsection{Attendence Exceptions}

All team members are expected to attend meetings. If a member will be arriving late or leaving early for any reason, it is their responsibility to communicate this beforehand and ask the team for a summary of what they missed. If a member cannot attend a meeting, they must present an acceptable excuse to the team before the meeting or ask for the meeting to be rescheduled.

\subsection{Acceptable Excuse}


Acceptable:\\
\begin {itemize}
    \item \textbf{Medical reasons} – personal illness, doctor appointments, or urgent medical issues.
    \item \textbf{Family emergency} – unexpected situations involving immediate family members, such as accidents or serious health issues.
    \item \textbf{Work-related conflicts} – overlapping meetings, urgent project deadlines, or unavoidable last-minute tasks. 
    \item \textbf{Technical difficulties} – internet or equipment failure preventing participation in virtual meetings.
    \item \textbf{Personal emergencies} – accidents, sudden household issues (e.g., plumbing, electricity), or car breakdowns.
    \item \textbf{Scheduled academic course} – a mandatory class or lecture that overlaps with the meeting time.
\end{itemize}


Not-acceptable:\\

\begin{itemize}
    \item Oversleeping or poor time management
    \item Forgetting the meeting or deadline
    \item Conflicting social plans or events
    \item Claiming ignorance of the meeting or deadline
    \item Being "too busy" with other tasks without prior communication
\end{itemize}

\subsection{In Case of Emergency}

\begin{itemize}
    \item Immediately notify the team
    \item Provide details about the situation and the expected impact on their availability
    \item If possible, share any work completed so far or delegate responsibilities to ensure the team can continue without delay
    \item Communicate updates on their availability as the situation progresses
\end{itemize}

\subsection{Accountability and Teamwork Quality}

Our team holds high expectations for the quality of work and preparation of meetings. Each member is expected to review relevant materials, contribute meaningful insights, and come prepared with suggestions and/or updates. In terms of deliverables, we prioritize accuracy, clarity, and adherence to any predefined guidelines or deadlines. Every contribution should reflect a high standard of professionalism, ensuring that it supports the team’s objectives and maintains the overall quality of our collaborative efforts.

\subsection{Attitude}

All team members should:

\begin{itemize}
    \item Be respectful of each other’s ideas and perspectives.
    \item Approach problems with an open mind; disagreements should be expressed respectfully and non-aggressively.
    \item Aim to contribute a similar amount of work as other team members.
    \item Ask for help on a task if needed; do not spend an unreasonable amount of time being stuck on a task without external support.
    \item Stay up to date with project progress and share updates on one’s own progress.
\end{itemize}

If a conflict arises with another member, those members should discuss it directly with the intention to resolve the issue cordially. If no resolution was achieved, the members must discuss the conflict with the team for the team’s opinions.

\subsection{Stay on Track}

\begin{itemize}
    \item GitHub Project Board and Issues: The team will use a GitHub project board to track tasks, assign issues, and monitor progress. Each member is expected to update their assigned tasks regularly and close issues when they are completed
    \item Development Plan: The team will follow the development plan outlined. Milestones, deadlines, and deliverables will be reviewed on the predetermined due dates
    \item Performance Management:
        \begin{itemize}
        \item Rewards: Members who consistently meet or exceed expectations will receive recognition within the team and maybe a free bubble tea from their favourite tea shop!
        \item Underperformance: Members who fail to contribute or consistently miss targets without valid reasons. Members can make up by taking on additional tasks or making up lost time.
        \item Consequences for Missing Targets: If a team member consistently fails to meet contribution targets, the team will discuss with the member to understand why this may be happening and offer support.
        \end{itemize}
\end{itemize}

\subsection{Team Building}

We will host a semi-competitive badminton tournament across members during McMaster Pulse drop-ins at least once per month. Other sport competitions may be discussed and considered, with possible suggestions being a maximum repetition bicep curl competition, a dynamic rock climbing move, and a maximum duration plank hold. We will also go out for bubble tea every other week to complain about our school workload and job applications.

\subsection{Decision Making}

In the event of a disagreement within the team, we will conduct a vote, and the decision will be based on the majority consensus. However, for critical decisions, a unanimous vote will be required from all team members before proceeding. The team will be consulted for all decisions so they may be thoroughly discussed. 

\end{document}
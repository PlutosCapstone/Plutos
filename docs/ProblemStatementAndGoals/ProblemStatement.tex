\documentclass{article}

\usepackage{tabularx}
\usepackage{booktabs}

\title{Problem Statement and Goals\\\progname}

\author{\authname}

\date{}

%% Comments

\usepackage{color}

\newif\ifcomments\commentstrue %displays comments
%\newif\ifcomments\commentsfalse %so that comments do not display

\ifcomments
\newcommand{\authornote}[3]{\textcolor{#1}{[#3 ---#2]}}
\newcommand{\todo}[1]{\textcolor{red}{[TODO: #1]}}
\else
\newcommand{\authornote}[3]{}
\newcommand{\todo}[1]{}
\fi

\newcommand{\wss}[1]{\authornote{blue}{SS}{#1}} 
\newcommand{\plt}[1]{\authornote{magenta}{TPLT}{#1}} %For explanation of the template
\newcommand{\an}[1]{\authornote{cyan}{Author}{#1}}

%% Common Parts

\newcommand{\progname}{ProgName} % PUT YOUR PROGRAM NAME HERE
\newcommand{\authname}{Team \#, Team Name
\\ Student 1 name
\\ Student 2 name
\\ Student 3 name
\\ Student 4 name} % AUTHOR NAMES                  

\usepackage{hyperref}
    \hypersetup{colorlinks=true, linkcolor=blue, citecolor=blue, filecolor=blue,
                urlcolor=blue, unicode=false}
    \urlstyle{same}
                                


\begin{document}

\maketitle

\begin{table}[hp]
\caption{Revision History} \label{TblRevisionHistory}
\begin{tabularx}{\textwidth}{llX}
\toprule
\textbf{Date} & \textbf{Developer(s)} & \textbf{Change}\\
\midrule
09/18/2024 & Angela Wang & Initial Draft\\
09/20/2024 & Jason Tan & Environment section and touch ups\\
09/24/2024 & Eric Chen & Update Appendix Reflection Questions\\
... & ... & ...\\
\bottomrule
\end{tabularx}
\end{table}

\section{Problem Statement}

% \wss{You should check your problem statement with the
% \href{https://github.com/smiths/capTemplate/blob/main/docs/Checklists/ProbState-Checklist.pdf}
% {problem statement checklist}.} 

% \wss{You can change the section headings, as long as you include the required
% information.}

\subsection{Problem}

Young adults often face challenges in managing their finances effectively,
especially when it comes to tracking expenses and budgeting. Despite
advancements in artificial intelligence (AI) and automation, many budgeting apps
still require manual data entry or calculations, resulting in an inefficient and
potentially inaccurate process. This inconvenience often leads users to poorly
manage their budgeting or abandon it altogether, hindering their ability to
optimize spending habits and achieve their financial goals.


\subsection{Inputs and Outputs}

% \wss{Characterize the problem in terms of ``high level'' inputs and outputs.  
% Use abstraction so that you can avoid details.}

\textbf{Inputs:} User's receipt photos and desired bugeting goals.\\
\textbf{Outputs:} Visualizations of the user's spending allocations in
comparison to their set budget, and recommendations for how they may adjust
their spending to meet their goals.

\subsection{Stakeholders}

Stakeholders include anyone who is looking to better manage their finances, set
budget goals, and track their spending, with a focus on first and second-year
university students who are just starting to live on their own.

\subsection{Environment}

% \wss{Hardware and software environment}

The software product will be compatible for Android and iOS mobile devices with
a functional camera.

\section{Goals}

Our goals include the following:
\begin{itemize}
    \item Develop a machine learning model that can accurately
    (\textgreater90\%) parse commonly purchased items (e.g., groceries, cleaning
    supplies) from a picture of a receipt.
    \item Develop a machine learning model that can accurately
    (\textgreater90\%) categorize items into approprite, pre-defined spending
    categories. 
    \item Develop a mobile application that allows users to take a picture of
    their receipt or manually input their expenses. These expenses would be
    stored in a database so that the user can review their spending history.
    \item Within the application, display visualizations of the user's purchases
    and spending allocations, and provide recommendations for how they may
    adjust their spending to meet their budget goals. These should be catered to
    the user's personal spending habits and goals.
    \item Allow users to set budget goals over different time intervals (e.g.,
    short-term, long-term) and track their progress towards these goals.
\end{itemize}

\section{Stretch Goals}
\begin{itemize}
    \item Build upon the base machine learning model to train on the user's
    personal spending data to provide more accurate item parsing and
    categorization (i.e., for items with similar names, the model can learn
    which category the user typically assigns them to).
    \item Build upon the base machine learning model to categorize items using
    user-defined/customizable spending categories.
    \item Build upon the base machine learning model to predict future spending
    based on the user's spending history and provide recommendations for how
    they can adjust their spending to meet their budget goals.
    \item Gamify the application to make it more engaging and encourate users to
    meet their budget goals and develop better spending habits.
    \item Allow users to input expenses through speech recognition, where the
    application can parse the user's speech and categorize the items
    accordingly.
\end{itemize}

\section{Challenge Level and Extras}

% \wss{State your expected challenge level (advanced, general or basic).  The
% challenge can come through the required domain knowledge, the implementation
% or something else.  Usually the greater the novelty of a project the greater
% its challenge level.  You should include your rationale for the selected
% level. Approval of the level will be part of the discussion with the
% instructor for approving the project.  The challenge level, with the approval
% (or request) of the instructor, can be modified over the course of the term.}

% \wss{Teams may wish to include extras as either potential bonus grades, or to
% make up for a less advanced challenge level.  Potential extras include
% usability testing, code walkthroughs, user documentation, formal proof,
% GenderMag personas, Design Thinking, etc.  Normally the maximum number of
% extras will be two.  Approval of the extras will be part of the discussion
% with the instructor for approving the project.  The extras, with the approval
% (or request) of the instructor, can be modified over the course of the term.}

The expected challenge level is \textbf{general}. The primary challenge of the
project is developing a machine learning model that can accurately parse items
from a picture of a receipt, and to categorize them into appropriate spending
categories. This requires a strong understanding of training and tuning models
on image data to achieve high accuracy. Additionally, different items across
various stores may have similar names or be difficult to recognize, which adds
to the complexity of the task. The other part of the project is to develop a
user-friendly mobile application, which is a more general software engineering
component.

Furthermore, our team plans to include the following two extras:
\begin{itemize}
    \item \textbf{Requirements elicitation}: We will conduct interviews and a
    survey to gather requirements from potential users to determine user needs
    and preferences.
    \item \textbf{Usability testing}: We will ask potential users to test the
    application and provide feedback on its usability and functionality.

\end{itemize}

\newpage{}


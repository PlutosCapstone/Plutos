\documentclass{article}

\usepackage{booktabs}
\usepackage{tabularx}
\usepackage{enumitem}
\usepackage{hyperref}

\title{Development Plan\\\progname}

\date{}

%% Comments

\usepackage{color}

\newif\ifcomments\commentstrue %displays comments
%\newif\ifcomments\commentsfalse %so that comments do not display

\ifcomments
\newcommand{\authornote}[3]{\textcolor{#1}{[#3 ---#2]}}
\newcommand{\todo}[1]{\textcolor{red}{[TODO: #1]}}
\else
\newcommand{\authornote}[3]{}
\newcommand{\todo}[1]{}
\fi

\newcommand{\wss}[1]{\authornote{blue}{SS}{#1}} 
\newcommand{\plt}[1]{\authornote{magenta}{TPLT}{#1}} %For explanation of the template
\newcommand{\an}[1]{\authornote{cyan}{Author}{#1}}

%% Common Parts

\newcommand{\progname}{ProgName} % PUT YOUR PROGRAM NAME HERE
\newcommand{\authname}{Team \#, Team Name
\\ Student 1 name
\\ Student 2 name
\\ Student 3 name
\\ Student 4 name} % AUTHOR NAMES                  

\usepackage{hyperref}
    \hypersetup{colorlinks=true, linkcolor=blue, citecolor=blue, filecolor=blue,
                urlcolor=blue, unicode=false}
    \urlstyle{same}
                                


\begin{document}

\maketitle

\begin{table}[hp]
\caption{Revision History} \label{TblRevisionHistory}
\begin{tabularx}{\textwidth}{llX}
\toprule
\textbf{Date} & \textbf{Developer(s)} & \textbf{Change}\\
\midrule
09/24/2024 & Payton Chan, Eric Chen, Fondson Lu & Revision 0\\
09/24/2024 & Jason Tan, Angela Wang & Revision 0\\
... & ... & ...\\
\bottomrule
\end{tabularx}
\end{table}

\newpage{}

% \wss{Put your introductory blurb here.  Often the blurb is a brief roadmap of
% what is contained in the report.}

% This document contains the development plan for Plutos.


\section{Confidential Information}
This project does not have confidential information from industry to protect.
However, we all personally identifiable information (PII) will be kept
confidential. Personal data will not be disclosed or made accessible to the
public, and is securely stored within each user's account, accessible only to
the account holder. We adhere to stringent security protocols to protect user
information in compliance with applicable data protection regulations.

\section{IP to Protect}
There is no Intellectual Property (IP) to protect.

\section{Copyright License}
We are adopting the MIT license, which enables the free distribution of our
work. The license can be found here:
\href{https://github.com/PlutosCapstone/Plutos/blob/main/LICENSE}{MIT License}.

\newpage

\section{Team Meeting Plan}

\subsection{Meeting Schedules and Locations}

\begin{itemize}
    \item Every \textbf{Monday} (excluding breaks \& holidays) from \textbf{2:30
    PM to 4:20 PM} (2 hours) at the designated tutorial time. All team members
    are required to attend, be punctual, and come prepared with updates or
    questions.
    
    \item The meeting will be held in person in one of the following buildings:
    \textbf{MDCL, LRW, Gerald Hatch Center, or Mills Library}. Consult the team
    the weekend before the meeting to choose a suitable tutorial/meeting room
    and if necessary, reserve it for a 2 hour time period.
    
    \item \textbf{Contingency for in-person meetings}: If extenuating
    circumstances (e.g., severe weather) prevent meeting in person, the meeting
    will be held virtually or postponed to the next available date based on
    group consensus. Meetings may also be held virtually if discussed and agreed
    upon by the majority of the members. Virtual meetings will take place on our
    shared Discord server.

    \item When required, the team can host additional meetings throughout the
    week that align with each member’s schedules to address certain
    roadblocks/issues encountered throughout the working week. The meeting will
    take place in the same buildings listed above or virtual if there is no
    seating available.
\end{itemize}

\subsection{Meeting Structure}
Each meeting will be scheduled in advance with an agenda prepared by the Meeting
Secretary (more details on roles discussed in Section 4.3.). Attendance will be
recorded at the start of each meeting, after which agenda items will be
discussed. A list of action items will be noted at the bottom of the agenda,
where each item will be assigned a specific team member for implementation. \\
\\
The weekly Monday meetings will be conducted in an agile/scrum-like approach. 
\begin{itemize}
    \item The first \textbf{15-30 minutes} will be dedicated to a standup, where
    each person will be giving a brief update on their progress for the prior
    week, their current plan/task/actions items for the upcoming week, and any
    roadblocks or questions they may have that concerns other parts of the
    system.
    \item The remaining time meeting is allotted for additional development
    work, peer programming and/or tech debt. Time should be used effectively
    based on the upcoming deadlines for that week.
\end{itemize}

\newpage

\section{Team Communication Plan}

\begin{itemize}
    \item The main source of communication between the team will be via
    \textbf{Discord}. Meetings that need to be conducted virtually will be done
    through the dedicated voice channel (\textit{General}) within the team
    server.
    
    \item All development-related issues will be hosted via the project’s GitHub
    Issues tab.
    \begin{itemize}
        \item This provides us a source of traceability for every task
        completion along with an interface to receive instructional feedback.
    \end{itemize}

    \item GitHub Projects will be used as our primary project management tool.
    This board will be reviewed every week during Monday meetings and regularly
    updated throughout the week.
\end{itemize}

\begin{table}[h!]
\centering
\caption{Team Contact Information}
\begin{tabular}{|l|l|l|l|}
\hline
\textbf{Member Name} & \textbf{Discord (primary contact)} & \textbf{MacID} &
\textbf{GitHub} \\ 
\hline
 Payton Chan   & @paytchan    & chanp29  & paytonchan \\ 
\hline
 Eric Chen     & @Er.\_.ic     & chene40  & chene40    \\ 
\hline 
Fondson Lu    & @chubbydukki & luh57    & fondsonlu  \\ 
\hline
Jason Tan     & @jason\_tan  & tanj60   & tan-jason  \\ 
\hline 
Angela Wang   & @ziyuna      & wanga91  & angelaw7   \\
\hline
\end{tabular}
\end{table}

\newpage

\section{Team Member Roles}
The roles listed below may be subject to change at any point throughout the
term, as deemed necessary to best accommodate the needs of the project and the
overall team objectives. They are currently unassigned; assignment of roles will
be done every few weeks during Monday meetings to ensure that everyone has a
chance at taking on certain responsibilities.

Additionally, these roles should not be seen as limiting the tasks or
responsibilities one can take on. Team members are expected to step up and
assume additional tasks when necessary, particularly when their expertise or
skills align with the project’s demands. It may also be necessary at certain
times for members to “wear multiple hats” and assume more than one role. \\
\\
\textbf{Meeting Chair}: The meeting chair’s job is to ensure the group meetings
are being conducted in an organized fashion and to facilitate the procedure for
internal conflicts. \\
\\
\textbf{Meeting Secretary}: The meeting secretary is responsible for preparing
meeting agendas and taking notes during all group discussions. These notes are
to be used for future reference and to keep track of thought processes that
should be provided in the project documentation. \\
\\
\textbf{Milestone Coordinator}: The milestone coordinator has the responsibility
of facilitating the creation and completion of deliverables. They are also
responsible for assuring that each deliverable meets the criteria provided and
that the standard of excellence is being maintained throughout the duration of
the project. \\
\\
\textbf{Documentation Coordinator}: The documentation coordinator reviews
documentation as the project progresses, ensuring that documents have a
consistent level of quality and reflects the current state of the project. They
work with team members to update, organize, and maintain a variety of documents,
while also implementing best practices and managing version control. \\
\\
\textbf{QA Tester}: The QA tester is responsible for evaluating software to
identify bugs and ensure it meets quality standards. They design and execute
test plans and collaborate with the development team to address issues, ensuring
a high-quality user experience. \\
\\
\textbf{Systems Architect}: The systems architect is responsible for designing
the overall architecture of a software system, ensuring that it meets both
technical and business requirements. \\
\\
\textbf{Frontend Specialist}: The frontend specialist focuses on creating the
user interface and user experience of a web application, using technologies like
HTML, CSS, and JavaScript. They work to ensure that the application is visually
appealing, responsive, and user-friendly, collaborating closely with designers
and backend developers. \\
\\
\textbf{Backend Specialist}: The backend specialist is responsible for
server-side logic, database interactions, and application functionality. They
build and maintain the core application infrastructure using programming
languages and frameworks, ensuring that the frontend and backend communicate
seamlessly to deliver a smooth and efficient user experience. \\
\\
\textbf{User Researcher}: The user researcher conducts studies to understand
user behaviors, needs, and motivations through methods like interviews, surveys,
and usability testing. They analyze the findings to provide insights that
enhance product design and development, ensuring that the final product aligns
with user expectations

\newpage

\section{Workflow Plan}

\subsection{GitHub Issues}
\begin{itemize}
    \item GitHub Issues will be used to track and manage all development tasks.
    \begin{itemize}
        \item An issue will be created for each task -- each task should be
        small enough such that one developer can finish it within a time frame
        of one week. If the task is too large, it should be broken down into
        smaller tasks.
        \item All issues must include the following fields:
        \begin{itemize}
            \item \textbf{Assignee(s)} - Indicates who has/will be working on
            the issue.
            \item \textbf{Labels} - Information regarding the category the
            ticket falls under (e.g., Feature, Bug, Tech Debt, Version Bump,
            etc.)
            \item \textbf{Milestone} - The milestone/deliverable that the ticket
            is associated with.
        \end{itemize}
    \end{itemize}
\end{itemize}

\subsection{GitHub Projects}
\begin{itemize}
    \item GitHub Projects will be used as the project management tool to manage
    the roadmap and plan the project.
    \begin{itemize}
        \item All issues should be placed in the appropriate column (Backlog, In
        Progress, In Review, Complete) and updated as soon as changes are made.
    \end{itemize}
\end{itemize}

\subsection{Git Branches}
\begin{itemize}
    \item Git Branches will be used to make changes to the project.
    \begin{itemize}
        \item The type of development work should be noted as a prefix to the
        branch
        \begin{itemize}
            \item \texttt{feat} for feature work.
            \item \texttt{bugfix} for bugfix.
            \item \texttt{chore} for chore or tech debt work.
        \end{itemize}
        \item Branches should follow the naming convention:
        \texttt{<prefix>/<modId>/<desc>...-<desc>}. This is to avoid conflicting
        branch names and to make it clear on the topic of the change.
        \begin{itemize}
            \item E.g., \texttt{feat/chene40/update-dependencies}.
        \end{itemize}
        \item Feature branches should always be branched off the updated main
        branch.
        \item Rebase branches whenever possible for pulling new updates.
        \item Once desired changes are complete on a branch, the author should
        open a pull request (PR) and follow the below PR guidelines.
    \end{itemize}
\end{itemize}

\newpage

\subsection{Pull Request (PR) Guidelines}
\begin{itemize}
    \item All PRs must include a descriptive title and a small description of
    what the PR achieved/changed.
    \item Upon opening a PR, the author(s) should inform the team that it is
    ready for review. One member of the team will be auto-assigned as a
    reviewer.
    \item All PRs must have at least one approved review before merging.
\end{itemize}

\begin{itemize}
    \item Reviewers should review a requested PR within a \textbf{24 hr} time
    frame (\textbf{48 hrs max}).
    \item PR should be listed as \texttt{Draft} if implementation/fixes have not
    been completed and converted to \texttt{Ready for Review} once a PR can be
    reviewed.
    \item PR must be linked to an issue using the appropriate \texttt{keywords}
    at the end of the descriptions section.
    \item All PRs are to be \textbf{squashed and merged}. This is important for
    ensuring the cleanliness of the commits on the main branch.
    \item On push to an open PR, a GitHub Action should run which deploys the
    branch to a CI/CD pipeline in which all automated tests should run to ensure
    at least 90\% line coverage. If this condition isn’t met, the PR will not be
    allowed to be merged.
\end{itemize}

\subsection{GitHub Actions}
\begin{itemize}
    \item GitHub Actions will be utilized for CI/CD to continuously deploy the
    application and keep it running at all times.
    \begin{itemize}
        \item Tests, linters, and formatters will be run on each push to a PR
        and the main branch.
        \item If a PR causes a GitHub Action to fail, it is the responsibility
        of the PR implementer and reviewer to resolve the issue as soon as
        possible so that it does not bottleneck other developers’ code
        integrations and implementations.
    \end{itemize}
\end{itemize}

\subsection{GitHub Tags}
\begin{itemize}
    \item GitHub Tags will be used to mark Revision 0 and Revision 1 versions of
    the project for each milestone. These should only be used for final versions
    of each revision and not draft work.
\end{itemize}

\newpage

\section{Project Decomposition and Scheduling}

\subsection{Project Management}
We will be using GitHub projects as our project management tool. It will be used
to organize and view Issues on a board organized by the status of the Issue
(i.e., Backlog, Ready, In progress, In review, Done), as well as to keep track
of the roadmap of the project. The project can be found here
\href{https://github.com/orgs/PlutosCapstone/projects/1/views/1}{GitHub
Project}.

\subsection{Project Decomposition}
The project will be decomposed into seven milestones over the duration of the
term. These milestones have provisional deadlines that may be subject to change
depending on the project's progress.

\subsubsection*{Milestone 0 - User Elicitation and Requirements Specification}
\begin{itemize}
    \item \textbf{Deadline:} Friday, October 4, 2024 @ 11.59 PM
    \item \textbf{Estimated Time:} 2 weeks
    \item All team members are expected to actively participate in distributing
    surveys and conducting in-person interviews across campus. The insights
    gathered from these user engagements will be crucial in defining and
    refining our project requirements, ensuring they are aligned with real user
    needs and expectations.
\end{itemize}

\subsubsection*{Milestone 1 - System Design, HLDs, DFDs, and POCs of Ploutos}
\begin{itemize}
    \item \textbf{Deadline:} Friday, October 4, 2024 @ 11.59 PM
    \item \textbf{Estimated Time:} 2 weeks
    \item All team members must convene to collaboratively discuss and refine
    the system design. This meeting will encompass the development of high-level
    designs (HLDs), data flow diagrams (DFDs), and proof of concepts (POCs). The
    objective is to ensure a cohesive architectural framework and to establish a
    well-defined workflow that will guide subsequent phases of the project.
\end{itemize}

\subsubsection*{Milestone 2 - Project Initialization \& Development Environment Setup}
\begin{itemize}
    \item \textbf{Deadline:} Friday, October 11, 2024 @ 11.59 PM (Note: Last Day
    Before Reading Week)
    \item \textbf{Estimated Time:} 1 week
    \item The work will be distributed evenly among team members, unless a
    consultation determines that another individual should take on additional
    responsibilities due to extenuating circumstances. The team member
    responsible for setting up the repository will also create the necessary
    onboarding documentation. Another individual will manage group-related
    accounts (e.g., AWS), while a designated member will oversee workflow
    management, including the creation of Git issue labels, milestones, and
    tasks. All team members are expected to follow the installation guide
    provided by the repository lead and ensure proper initialization of the
    development environment.
\end{itemize}

\subsubsection*{Milestone 3 - Create Preliminary Figma Designs for Mobile App Workflow}
\begin{itemize}
    \item \textbf{Deadline:} Friday, November 1, 2024 @ 11.59 PM (Note: Due 2
    Weeks After Reading Week)
    \item \textbf{Estimated Time:} 2 weeks
    \item All team members are expected to contribute to a collaborative Figma
    file to develop the necessary mockups and wireframes. Individuals with more
    experience in Figma may assume leadership roles to facilitate the design
    process, while those with less familiarity are encouraged to enhance their
    skills through active participation. It is essential that every team member
    provides valuable input, ensuring that the designs reflect a comprehensive
    and cohesive approach to the project’s objectives.
\end{itemize}

\subsubsection*{Milestone 4 - Translate and Implement Figma Wireframes Into Front End Code}
\begin{itemize}
    \item \textbf{Deadline:} Sunday, December 1, 2024 @ 11.59 PM (Estimated
    Time: 3 Weeks Excluding Exam Season and Winter Break)
    \item \textbf{Estimated Time:} 4 weeks
    \item All team members are expected to collaborate in translating the Figma
    wireframes into functional front-end code. Individuals with more experience
    in front-end development may take on leadership roles to guide the
    implementation process, while those with less familiarity are encouraged to
    deepen their understanding through active involvement. It is imperative that
    each team member contributes their insights and expertise, ensuring that the
    final product adheres to the design specifications and meets the project’s
    quality standards.
\end{itemize}

\subsubsection*{Milestone 5 - Build Out Back End API Services and Connect with Front End}
\begin{itemize}
    \item \textbf{Deadline:} Friday, February 1, 2025 @ 11.59 PM (Note: Exam
    Season Starts; Estimated Time: 3 Weeks)
    \item \textbf{Estimated Time:} 4 weeks
    \item All team members are expected to collaborate in developing the
    necessary back-end API services and ensuring their integration with the
    front-end components. Those with more experience in back-end development may
    assume leadership roles to oversee the implementation process, while less
    experienced members are encouraged to enhance their skills through active
    participation. It is crucial that every team member contributes their
    knowledge and insights, ensuring that the final services are robust,
    efficient, and aligned with the overall project objectives.
\end{itemize}

\subsubsection*{Milestone 6 - E2E Test, Cleanup, Final Deploy and Showcase for Expo}
\begin{itemize}
    \item \textbf{Deadline:} Friday, February 15, 2025 @ 11.59 PM
    \item \textbf{Estimated Time:} 2 weeks
    \item All team members are expected to collaborate in performing end-to-end
    testing as a cohesive unit. This collaborative effort will provide an
    opportunity to identify and resolve any lingering bugs, refine necessary
    documentation, and collectively prepare for the Expo showcase. It is
    imperative that each member contributes their insights and expertise,
    ensuring that the final product is polished and ready for presentation.
\end{itemize}

\subsubsection*{Milestone 7 - Final Documentation}
\begin{itemize}
    \item \textbf{Deadline:} Wednesday, April 2, 2025 @ 11.59 PM
    \item \textbf{Estimated Time:} 2 weeks (Start March 16, 2025 or earlier)
    \item It is crucial that the project team engages in a collaborative effort
    to aggregate and meticulously refine all pertinent documents, including the
    user guide, requirements, and design documentation, approximately two weeks
    prior to the submission deadline. This coordinated approach will facilitate
    thorough review and revision, ensuring that all materials are cohesive,
    accurately reflect the project’s objectives, and adhere to the highest
    standards of quality. Such diligence will be paramount for a successful
    final submission.
\end{itemize}

\newpage

\subsubsection*{Milestone 3-6 - Integrate and Train Optical Character Recognition (OCR) Model Into Scanner (To Be Completed Concurrently Starting Milestone 3)}
\begin{itemize}
    \item \textbf{Training Period:} October 21, 2024 - March 21, 2025
    \item All team members are required to actively engage in the training of
    our OCR machine learning model. Each member must collect and retain receipts
    as inputs for the model whenever feasible, as these inputs are critical for
    facilitating continuous testing and refinement. These receipts should be
    uploaded to the shared Drive folder. This iterative training process is to
    be conducted concurrently with other development activities for the
    application, ensuring that the model not only evolves in response to
    real-world data but also integrates seamlessly with the overall project. By
    committing to this collaborative effort, the team will enhance the model’s
    accuracy and reliability, ultimately contributing to the success of the
    application.
\end{itemize}

\newpage

\section{Proof of Concept Demonstration Plan}

\subsection{Plan for Proof of Concept Demonstration in November}
Demonstrate a working mobile application on a mobile device that can perform the
following functionalities:
\begin{enumerate}
    \item When a button is pressed, the phone camera opens. The user can then
    take a picture of an itemized receipt.
    \item Once the picture has been taken, the application runs it through the
    ML model, where it processes the image. The application should then list out
    the recognized items from the receipt. Items should also be tagged or
    grouped into predefined categories.
\end{enumerate}

\subsection{Anticipated Risks and Mitigation Plan}

\begin{itemize}
    \item \textbf{Data Accuracy and Reliability:} The ML model might
    misinterpret or misclassify data from receipts, leading to incorrect
    financial records.
    \begin{itemize}
        \item \textbf{Risk level:} High, as it might take more data than
        anticipated to train a model to recognize and classify items
        appropriately.
        \item \textbf{Mitigation plan:} We will train the model using a large
        dataset of receipts that contains various items from different
        categories, and then test the model to verify the accuracy of the
        results. An alternative option to manually edit the price, category,
        date/time, and other metadata fields about the items recorded will be
        provided, and this would be used as feedback to improve the model.
    \end{itemize}
    
    \item \textbf{Model Generalization and Adaptability:} The ML model may
    struggle with adapting to new types of receipts, retailers, or products that
    it hasn’t encountered during training, resulting in misclassifications.
    \begin{itemize}
        \item \textbf{Risk level:} Low, in terms of the POC demonstration. We
        plan to demonstrate using a receipt that lists commonly purchased items
        for the purpose of showing the feasibility of the model.
        \item \textbf{Mitigation plan:} For the first iteration of the model, we
        plan to target specific stores near campus (e.g., Food Basics, Fortinos,
        Shoppers Drug Mart) as we will be able to collect the most amount of
        data from these locations and monitor the accuracy of the model. As the
        project progresses, we will expand to different categories and stores.
    \end{itemize}
    
    \item \textbf{Optical Character Recognition (OCR) Limitations:} Poor-quality
    receipts (faded text, creases, or low resolution) may hinder OCR
    performance, leading to incomplete or inaccurate data extraction.
     \begin{itemize}
         \item \textbf{Risk level:} Medium, as we plan to use a good quality
         receipt (i.e., laid out flat on a table, newly printed) for the POC
         demonstration.
        \item \textbf{Mitigation plan:} We will initially train the model using
        good quality receipts. Over the course of the training process, we plan
        to introduce lower-quality receipts and use advanced pre-processing
        techniques (e.g., noise reduction, image enhancement) to increase the
        accuracy of parsing lower-quality data. For the POC demonstration, we
        plan to use a good quality receipt for the purpose of showing the base
        functionality of the model.
      \end{itemize}

    \item \textbf{Privacy and Data Security:} Handling sensitive financial data
    (such as purchase histories and personal information) poses the risk of data
    breaches or unauthorized access.
    \begin{itemize}
        \item \textbf{Risk level:} Low, as we do not plan to release the
        application before the POC. We will only be testing locally with our
        team of developers, so there would be no unauthorized access.
         \item \textbf{Mitigation Plan:} To mitigate privacy and data security
         risks during the POC, sensitive financial data will be encrypted both
         at rest and in transit, with access restricted to authorized developers
         using role-based controls. Anonymized or synthetic data will be used
         for testing, and local development environments will be hardened with
         strong passwords, disk encryption, and up-to-date software. 
    \end{itemize}

    \item \textbf{User Experience:} If the app struggles with receipt clarity,
    formatting inconsistencies, or unusual items, users might become frustrated.
    \begin{itemize}
        \item \textbf{Risk level:} Low, as we plan to make user experience a
        priority.
        \item \textbf{Mitigation plan:} We will conduct usability testing as
        part of our project, where users will test the app and give feedback on
        points of improvement for the receipt scanning workflow. For the POC
        demonstration, we plan to show the key functionality of taking a photo
        of a receipt and displaying the outputs of the model on the screen.
        Wireframes will be designed prior to development to ensure that the user
        interface and experience is well thought out.
    \end{itemize}
    
    \item \textbf{Scalability and Performance:} As the user base grows, handling
    a large volume of receipt uploads and processing requests in real-time could
    impact system performance and user experience.
    \begin{itemize}
        \item \textbf{Risk level:} Medium, in terms of performance for the POC
        plan. We are unsure how fast the model would be able to process an image
        and return the information that we would need. Scalability is not a risk
        for the POC plan but is important to consider for future development.
        \item \textbf{Mitigation plan:} We will run load testing and stress
        testing simulations to show the app’s capacity to handle increasing
        numbers of users and transactions without degradation in response time
        or accuracy. This does not apply for the POC demonstration but is
        important to consider for future development.
    \end{itemize}
\end{itemize}

\newpage

\section{Expected Technology}

The following lists the technology that we expect or are considering to use for
the project. Note that these are subject to change over the course of the
project.

\begin{itemize}
    \item \textbf{Programming languages}
    \begin{itemize}
        \item Frontend: React Native, TypeScript, Dart/Flutter
        \item Backend: Python, Golang, AWS
    \end{itemize}
    
    \item \textbf{Libraries}
    \begin{itemize}
        \item Utility CSS Library: Tailwind, MUI, Bootstrap, Css-in-js, Styled
        Components
        \item Authentication: Auth0, Firebase
    \end{itemize}
    
    \item \textbf{Tools}
    \begin{itemize}
        \item Database: MongoDB, Firebase, Amazon DynamoDB
    \end{itemize}
    
    \item \textbf{OCR models}
    \begin{itemize}
        \item Tesseract OCR
        \item Google Vision API
        \item Amazon Textract
    \end{itemize}
    
    \item \textbf{Linter tools}
    \begin{itemize}
        \item eslint
        \item typescript-eslint
    \end{itemize}
    
    \item \textbf{Testing frameworks}
    \begin{itemize}
        \item Jest (Unit)
        \item Mocha/Chai (Unit)
        \item React Native Testing Library (Unit + Integration)
        \item Detox (E2E)
        \item Maestro (UI \& Flow Tests)
    \end{itemize}
    
    \item \textbf{Continuous integration (CI)}
    \begin{itemize}
        \item GitHub Actions for running tests, linting, formatting. More
        details on CI plan can be found in Section 7.
    \end{itemize}
    
    \item \textbf{Version control}
    \begin{itemize}
        \item Git, GitHub
    \end{itemize}
    
    \item \textbf{Project management}
    \begin{itemize}
        \item GitHub projects
    \end{itemize}
\end{itemize}

\section{Coding Style/Standards}

\begin{itemize}
    \item Most of the code styling will be done via third party formatters and
    linters such as Prettier and ESLint. These formatters/linters will be
    implemented directly into the project itself via Git Hooks and will be
    triggered on every pre-commit via \textbf{Husky}.
    \item As a basis for committing changes, every commit should be atomic and
    contain one change pertaining to a specific feature/fix.
    \item \textbf{Commit Messages:}
    \begin{itemize}
        \item All commit messages should be systematic and descriptive.
        \item All commit messages should be written in present imperative tense
        \begin{itemize}
            \item \textbf{Correct:} "Fix typo in README"
            \item \textbf{Incorrect:} "Fixed typo in README"
        \end{itemize}
        \item All commit messages should begin with one of the following
        prefixes:
        \begin{itemize}
            \item \texttt{feat:} - for feature and enhancements related
            implementations
            \item \texttt{fix:} - for bug fixes and tech debt related fixes
            \item \texttt{chore:} - for project related commits such as bumping
            project/dependency versions, file renames/relocations, etc.
        \end{itemize}
        \item Example commit messages:
        \begin{itemize}
            \item \texttt{feat: generate openapi ts client based on
            api\_spec.yaml}
            \item \texttt{fix: change color of CTA button from red to blue}
            \item \texttt{chore: reorganize file structure hierarchy}
        \end{itemize}
    \end{itemize}
    \item We will use PEP 8 style guides for Python code
\end{itemize}

\newpage{}

\section{Appendix --- Reflection}

\begin {enumerate}
\item \textbf{Why is it important to create a development plan prior to starting the
Project?}

Creating a development plan before starting the project is crucial so the whole team can discuss/agree upon the key project details and scope. It is vital to lay out the groundwork for our project and define the direction needed to achieve our goals optimally. We found it especially important to discuss team dynamics – specifying meeting details, expectations from each member of the team, and the general workflow plan. Defining these elements upfront helps keep everyone aligned and organized before we dive into the project details and implementation. Breaking down the project into high-level milestones allows us to create well-defined and achievable timelines to guide our progress.

\item \textbf{In your opinion, what are the advantages and disadvantages of using CI/CD?}

We believe that CI/CD is a great tool due to it allowing automation and quality control within our development workflow. This significantly reduces manual testing labour and saves time which will be crucial for the development of the Plutos app. We aim to at least include running tests, linters, and formatters within our CI pipeline so that we may be confident that all changes made meet a certain quality level. One drawback that we will need to be aware of is that the setup of the CI/CD pipeline may pose a challenge due to the team’s lack of experience in setting up such a workflow. Having too much automation could lead us to have false confidence in our code, especially if our test coverage is not meeting standards due to rapid development. It’s important not to rely too heavily on the CI/CD pipeline and understand that it’s a tool that is meant to help gauge the overall quality of our code and not something that will replace manual testing.

\item \textbf{What disagreements did your group have in this deliverable, if any, and how did you resolve them?}

Our team had differing opinions when we discussed the timeline of our project. Some members felt we could reduce the time allocated for end-to-end testing from five weeks to three, while others preferred to extend the frontend and backend development by an additional week. We also had discussions for how much buffer time was needed to account for potential delays. After some discussion, we came to an agreement that the frontend and backend could be done in parallel, allowing us to dedicate 6 weeks to both sections. These differences were ultimately resolved through mutual understanding, as many of us would be unavailable during exam season in December and January. What ultimately helped us resolve our differences was the fact that we all remained realistic about potential challenges we might face in the future.

\end{enumerate}

\section{Appendix --- Team Charter}

\subsection{External Goals}

The team's objectives are to gain proficiency in AI/ML and to develop a well functioning application to showcase (and for people to try out) at the Capstone Expo. We also aim for the code and application to be clean and presentable for potential interviews.

\subsection{Attendence Exceptions}

All team members are expected to attend meetings. If a member will be arriving late or leaving early for any reason, it is their responsibility to communicate this beforehand and ask the team for a summary of what they missed. If a member cannot attend a meeting, they must present an acceptable excuse to the team before the meeting or ask for the meeting to be rescheduled.

\subsection{Acceptable Excuse}


Acceptable:\\
\begin {itemize}
    \item \textbf{Medical reasons} – personal illness, doctor appointments, or urgent medical issues.
    \item \textbf{Family emergency} – unexpected situations involving immediate family members, such as accidents or serious health issues.
    \item \textbf{Work-related conflicts} – overlapping meetings, urgent project deadlines, or unavoidable last-minute tasks. 
    \item \textbf{Technical difficulties} – internet or equipment failure preventing participation in virtual meetings.
    \item \textbf{Personal emergencies} – accidents, sudden household issues (e.g., plumbing, electricity), or car breakdowns.
    \item \textbf{Scheduled academic course} – a mandatory class or lecture that overlaps with the meeting time.
\end{itemize}


Not-acceptable:\\

\begin{itemize}
    \item Oversleeping or poor time management
    \item Forgetting the meeting or deadline
    \item Conflicting social plans or events
    \item Claiming ignorance of the meeting or deadline
    \item Being "too busy" with other tasks without prior communication
\end{itemize}

\subsection{In Case of Emergency}

\begin{itemize}
    \item Immediately notify the team
    \item Provide details about the situation and the expected impact on their availability
    \item If possible, share any work completed so far or delegate responsibilities to ensure the team can continue without delay
    \item Communicate updates on their availability as the situation progresses
\end{itemize}

\subsection{Accountability and Teamwork Quality}

Our team holds high expectations for the quality of work and preparation of meetings. Each member is expected to review relevant materials, contribute meaningful insights, and come prepared with suggestions and/or updates. In terms of deliverables, we prioritize accuracy, clarity, and adherence to any predefined guidelines or deadlines. Every contribution should reflect a high standard of professionalism, ensuring that it supports the team’s objectives and maintains the overall quality of our collaborative efforts.

\subsection{Attitude}

All team members should:

\begin{itemize}
    \item Be respectful of each other’s ideas and perspectives.
    \item Approach problems with an open mind; disagreements should be expressed respectfully and non-aggressively.
    \item Aim to contribute a similar amount of work as other team members.
    \item Ask for help on a task if needed; do not spend an unreasonable amount of time being stuck on a task without external support.
    \item Stay up to date with project progress and share updates on one’s own progress.
\end{itemize}

If a conflict arises with another member, those members should discuss it directly with the intention to resolve the issue cordially. If no resolution was achieved, the members must discuss the conflict with the team for the team’s opinions.

\subsection{Stay on Track}

\begin{itemize}
    \item GitHub Project Board and Issues: The team will use a GitHub project board to track tasks, assign issues, and monitor progress. Each member is expected to update their assigned tasks regularly and close issues when they are completed
    \item Development Plan: The team will follow the development plan outlined. Milestones, deadlines, and deliverables will be reviewed on the predetermined due dates
    \item Performance Management:
        \begin{itemize}
        \item Rewards: Members who consistently meet or exceed expectations will receive recognition within the team and maybe a free bubble tea from their favourite tea shop!
        \item Underperformance: Members who fail to contribute or consistently miss targets without valid reasons. Members can make up by taking on additional tasks or making up lost time.
        \item Consequences for Missing Targets: If a team member consistently fails to meet contribution targets, the team will discuss with the member to understand why this may be happening and offer support.
        \end{itemize}
\end{itemize}

\subsection{Team Building}

We will host a semi-competitive badminton tournament across members during McMaster Pulse drop-ins at least once per month. Other sport competitions may be discussed and considered, with possible suggestions being a maximum repetition bicep curl competition, a dynamic rock climbing move, and a maximum duration plank hold. We will also go out for bubble tea every other week to complain about our school workload and job applications.

\subsection{Decision Making}

In the event of a disagreement within the team, we will conduct a vote, and the decision will be based on the majority consensus. However, for critical decisions, a unanimous vote will be required from all team members before proceeding. The team will be consulted for all decisions so they may be thoroughly discussed. 

\end{document}

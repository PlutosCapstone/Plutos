\documentclass{article}

\usepackage{float}
\restylefloat{table}

\usepackage{booktabs}

\title{Team Contributions: POC\\\progname}

\author{\authname}

\date{}

%% Comments

\usepackage{color}

\newif\ifcomments\commentstrue %displays comments
%\newif\ifcomments\commentsfalse %so that comments do not display

\ifcomments
\newcommand{\authornote}[3]{\textcolor{#1}{[#3 ---#2]}}
\newcommand{\todo}[1]{\textcolor{red}{[TODO: #1]}}
\else
\newcommand{\authornote}[3]{}
\newcommand{\todo}[1]{}
\fi

\newcommand{\wss}[1]{\authornote{blue}{SS}{#1}} 
\newcommand{\plt}[1]{\authornote{magenta}{TPLT}{#1}} %For explanation of the template
\newcommand{\an}[1]{\authornote{cyan}{Author}{#1}}

%% Common Parts

\newcommand{\progname}{ProgName} % PUT YOUR PROGRAM NAME HERE
\newcommand{\authname}{Team \#, Team Name
\\ Student 1 name
\\ Student 2 name
\\ Student 3 name
\\ Student 4 name} % AUTHOR NAMES                  

\usepackage{hyperref}
    \hypersetup{colorlinks=true, linkcolor=blue, citecolor=blue, filecolor=blue,
                urlcolor=blue, unicode=false}
    \urlstyle{same}
                                


\begin{document}

\maketitle

This document summarizes the contributions of each team member up to the POC
Demo.  The time period of interest is the time between the beginning of the term
and the POC demo.

\section{Demo Plans}

We will be demonstrating a working prototype of the Plutos application, with the
following base functionalities:
\begin{itemize}
    \item Upload a receipt image
    \item Pass the image through the machine learning (ML) model to extract
    information and categorize items
    \item Display the extracted information to the screen
\end{itemize}

These features would serve as the base for the product and to demonstrate the
feasibility of the project.

\section{Team Meeting Attendance}

\begin{table}[H]
\centering
\begin{tabular}{ll}
\toprule
\textbf{Student} & \textbf{Meetings}\\
\midrule
\textbf{Total} & \textbf{1}\\
Payton Chan & 1\\
Eric Chen & 1\\
Fondson Lu & 1\\
Jason Tan & 1\\
Angela Wang & 1\\
\bottomrule
\end{tabular}
\end{table}

We only had \href{https://github.com/PlutosCapstone/Plutos/issues/9}{one team
meeting} to brainstorm the project idea. All other communication has been done
asynchronously through our Discord server or informally when we saw each other
during the week.

\section{Supervisor/Stakeholder Meeting Attendance}

\begin{table}[H]
\centering
\begin{tabular}{ll}
\toprule
\textbf{Student} & \textbf{Meetings}\\
\midrule
\textbf{Total} & \textbf{0}\\
Payton Chan & 0\\
Eric Chen & 0\\
Fondson Lu & 0\\
Jason Tan & 0\\
Angela Wang & 0\\
\bottomrule
\end{tabular}
\end{table}

We do not have a supervisor for our project. Additionally, we conducted a survey
for requirements elicitation, which did not require any direct meetings with
stakeholders. 

\section{Lecture Attendance}

\begin{table}[H]
\centering
\begin{tabular}{ll}
\toprule
\textbf{Student} & \textbf{Lectures}\\
\midrule
\textbf{Total} & \textbf{6}\\
Payton Chan & 6\\
Eric Chen & 6\\
Fondson Lu & 6\\
Jason Tan & 6\\
Angela Wang & 6\\
\bottomrule
\end{tabular}
\end{table}


\section{TA Document Discussion Attendance}

\begin{table}[H]
\centering
\begin{tabular}{ll}
\toprule
\textbf{Student} & \textbf{Lectures}\\
\midrule
\textbf{Total} & \textbf{3}\\
Payton Chan & 3\\
Eric Chen & 3\\
Fondson Lu & 3\\
Jason Tan & 3\\
Angela Wang & 3\\
\bottomrule
\end{tabular}
\end{table}

A record of the meetings can be found
\href{https://github.com/PlutosCapstone/Plutos/issues?q=label%3Ameeting+is%3Aclosed}{here}.



\section{Commits}

\begin{table}[H]
\centering
\begin{tabular}{lll}
\toprule
\textbf{Student} & \textbf{Commits} & \textbf{Percent (\%)}\\
\midrule
\textbf{Total} & \textbf{67} & \textbf{100} \\
Payton Chan & 8 & 11.94\\
Eric Chen & 11 & 16.42\\
Fondson Lu & 4 & 5.97\\
Jason Tan & 14 & 20.90\\
Angela Wang & 30 & 44.78\\
\bottomrule
\end{tabular}
\end{table}

Up until now, we have only worked on documentation, where we have decided to
make most of the changes on a Google Doc and then transfer it over to the LaTeX
document when sections are reviewed and mostly finalized. We chose to do this to
avoid merge conflicts and to make it easier to review, especially when multiple
members are working at the same time.

\section{Issue Tracker}

\begin{table}[H]
\centering
\begin{tabular}{lll}
\toprule
\textbf{Student} & \textbf{Authored (O+C)} & \textbf{Assigned (C only)}\\
\midrule
Payton Chan & 0 & 5\\
Eric Chen & 1 & 5\\
Fondson Lu & 1 & 5\\
Jason Tan & 5 & 5\\
Angela Wang & 14 & 5\\
\bottomrule
\end{tabular}
\end{table}

Up until this point, we have only worked on documentation. Angela has been the
milestone coordinator and meeting secretary up to the POC demo, so she has been
responsible for creating most of the issues and assigning them to the team
members. More about these roles can be found in Section 5 of the
\href{https://github.com/PlutosCapstone/Plutos/blob/main/docs/DevelopmentPlan/DevelopmentPlan.pdf}{Development
Plan}.

Additionally, since this is only documentation work, we have collectively agreed
that it was not necessary to create issues for the different sections of the
document. Instead, everyone was responsible for working on incomplete sections,
making changes when appropriate, and reviewing the document as a whole. This
method enables everyone to have a shared understanding and alignment of the
entire document.

\section{CICD}

Continuous integration and continuous deployment (CI/CD) will be used to
automatically test and deploy the application. The pipeline will be implemented
using GitHub Actions.

\end{document}
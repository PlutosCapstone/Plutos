\documentclass{article}

\usepackage{float}
\restylefloat{table}

\usepackage{booktabs}

\title{Team Contributions: Rev 0\\\progname}

\author{\authname}

\date{}

%% Comments

\usepackage{color}

\newif\ifcomments\commentstrue %displays comments
%\newif\ifcomments\commentsfalse %so that comments do not display

\ifcomments
\newcommand{\authornote}[3]{\textcolor{#1}{[#3 ---#2]}}
\newcommand{\todo}[1]{\textcolor{red}{[TODO: #1]}}
\else
\newcommand{\authornote}[3]{}
\newcommand{\todo}[1]{}
\fi

\newcommand{\wss}[1]{\authornote{blue}{SS}{#1}} 
\newcommand{\plt}[1]{\authornote{magenta}{TPLT}{#1}} %For explanation of the template
\newcommand{\an}[1]{\authornote{cyan}{Author}{#1}}

%% Common Parts

\newcommand{\progname}{ProgName} % PUT YOUR PROGRAM NAME HERE
\newcommand{\authname}{Team \#, Team Name
\\ Student 1 name
\\ Student 2 name
\\ Student 3 name
\\ Student 4 name} % AUTHOR NAMES                  

\usepackage{hyperref}
    \hypersetup{colorlinks=true, linkcolor=blue, citecolor=blue, filecolor=blue,
                urlcolor=blue, unicode=false}
    \urlstyle{same}
                                


\begin{document}

\maketitle

This document summarizes the contributions of each team member for the Rev 0
Demo.  The time period of interest is the time between the POC demo and the Rev
0 demo.

\section{Demo Plans}

We will be demonstrating a working version of the Plutos application, with the
following functionalities:
\begin{itemize}
    \item The user will upload a receipt image from a mobile device
    \item The image will be passed through the optical character recognition
    (OCR) library to parse text from the image, and then through the machine
    learning (ML) model to categorize the items
    \item The categorized items will be displayed to the user
    \item The data and the receipt image will be stored under the user's account
    \item The user will be able to view their past expenses and trends over time
    \item The user will be able to set up budgets for each spending category,
    and view their spending in relation to these budgets
\end{itemize}

\noindent These are the features that make up the core of the Plutos application.

\section{Team Meeting Attendance}


\begin{table}[H]
\centering
\begin{tabular}{ll}
\toprule
\textbf{Student} & \textbf{Meetings}\\
\midrule
\textbf{Total} & \textbf{5}\\
Payton Chan & 5\\
Eric Chen & 5\\
Fondson Lu & 5\\
Jason Tan & 5\\
Angela Wang & 5\\
\bottomrule
\end{tabular}
\end{table}

We only had one team meeting last term, and received feedback that structured
meetings may be useful over asynchronous communication or unstructured ones
(i.e., when we see each other during class). Therefore, we introduced more
meetings (about one every other week) this term to discuss progress and updates.

\section{Supervisor/Stakeholder Meeting Attendance}

\begin{table}[H]
    \centering
    \begin{tabular}{ll}
    \toprule
    \textbf{Student} & \textbf{Meetings}\\
    \midrule
    \textbf{Total} & \textbf{0}\\
    Payton Chan & 0\\
    Eric Chen & 0\\
    Fondson Lu & 0\\
    Jason Tan & 0\\
    Angela Wang & 0\\
    \bottomrule
    \end{tabular}
    \end{table}
    
We do not have a supervisor for our project, and we have not conducted usability
testing with stakeholders yet.

\section{Lecture Attendance}

\begin{table}[H]
    \centering
    \begin{tabular}{ll}
    \toprule
    \textbf{Student} & \textbf{Lectures}\\
    \midrule
    \textbf{Total} & \textbf{7}\\
    Payton Chan & 7\\
    Eric Chen & 7\\
    Fondson Lu & 7\\
    Jason Tan & 7\\
    Angela Wang & 7\\
    \bottomrule
    \end{tabular}
\end{table}

\section{TA Document Discussion Attendance}


\begin{table}[H]
    \centering
    \begin{tabular}{ll}
    \toprule
    \textbf{Student} & \textbf{Lectures}\\
    \midrule
    \textbf{Total} & \textbf{4}\\
    Payton Chan & 4\\
    Eric Chen & 4\\
    Fondson Lu & 4\\
    Jason Tan & 4\\
    Angela Wang & 4\\
    \bottomrule
    \end{tabular}
\end{table}

A record of the meetings can be found
\href{https://github.com/PlutosCapstone/Plutos/issues?q=label%3Ameeting+is%3Aclosed}{here}.
    

\section{Commits}

\wss{For each team member how many commits to the main branch have been made
over the time period of interest.  The total is the total number of commits for
the entire team since the beginning of the term.  The percentage is the
percentage of the total commits made by each team member.}

\begin{table}[H]
\centering
\begin{tabular}{lll}
\toprule
\textbf{Student} & \textbf{Commits} & \textbf{Percent}\\
\midrule
Total & Num & 100\% \\
Name 1 & Num & \% \\
Name 2 & Num & \% \\
Name 3 & Num & \% \\
Name 4 & Num & \% \\
Name 5 & Num & \% \\
\bottomrule
\end{tabular}
\end{table}

\wss{If needed, an explanation for the counts can be provided here.  For
instance, if a team member has more commits to unmerged branches, these numbers
can be provided here.  If multiple people contribute to a commit, git allows for
multi-author commits.}

\section{Issue Tracker}

\wss{For each team member how many issues have they authored (including open and
closed issues (O+C)) and how many have they been assigned (only counting closed
issues (C only)) over the time period of interest.}

\begin{table}[H]
\centering
\begin{tabular}{lll}
\toprule
\textbf{Student} & \textbf{Authored (O+C)} & \textbf{Assigned (C only)}\\
\midrule
Name 1 & Num & Num \\
Name 2 & Num & Num \\
Name 3 & Num & Num \\
Name 4 & Num & Num \\
Name 5 & Num & Num \\
\bottomrule
\end{tabular}
\end{table}

\wss{If needed, an explanation for the counts can be provided here.}

\section{CICD}

Continuous integration and continuous deployment (CI/CD) has been set up to do the following:
\begin{itemize}
    \item Lint and test the frontend code (if there are \texttt{src/client}
    changes)
    \item Compile and generate PDFs from the LaTeX code (if there are
    \texttt{docs/} changes)
\end{itemize}

\end{document}